\documentclass[a4paper,11pt]{report}
\usepackage[a4paper]{geometry}
\usepackage{graphicx}
\usepackage{amssymb}
\usepackage{epstopdf}
\usepackage{longtable,lscape,multirow}
\usepackage{showresults}
\usepackage[pdftex]{hyperref}
\usepackage[split]{splitidx}
\makeindex
\newindex[Beef cattle name index]{beef}
\newindex[Sheep name index]{sheep}
\newindex[Horse name index]{horse}
\newindex[Exhibitor index]{exhibitor}
% To create the index, run this file, then run the file `BatchMakeIndex` from the command line, then rerun this file

\DeclareGraphicsRule{.tif}{png}{.png}{`convert #1 `basename #1 .tif`.png}

%\textwidth = 6.5 in
%\textheight = 9 in
%\oddsidemargin = 0.0 in
%\evensidemargin = 0.0 in
%\topmargin = 0.0 in
%\headheight = 0.0 in
%\headsep = 0.0 in
%\parskip = 0.2in
%\parindent = 0.0in

%\includeonly{Dounby/Main,County/Main}

\title{Orkney Show Results $1936 - 2011$}
\author{Robert Scarth}
\begin{document}
	\maketitle
	\newpage
	
	\vbox to \textheight{ \vfil
	\begin{center}
	``...my work consists of two parts: of the one which is here, and of everything which I have not written. And precisely this second part is the important one.'' \\
	 - Ludwig Wittgenstein, letter to Ludwig von Ficker, 1919
	\end{center}
	\vfil}
	
	\newpage
	
		\tableofcontents
	
	\newpage

		\listoffigures
		
	\newpage

		
	I compiled these tables of previous prize winners at the Orkney shows from old show reports in \emph{The Orcadian}, and \emph{Orkney Today}. They probably contain errors. If you spot any errors I'd be grateful if you could let me know by contacting me through the facebook page ``Old Orkney Show Results'' here:\\ \href{https://www.facebook.com/Old-Orkney-Show-Results-307803273016114/}{\texttt{https://www.facebook.com/Old-Orkney-Show-Results-307803273016114/}} or at the email \texttt{orkneyshowresults@gmail.com}. You can download the latest version of this document from\\ \href{https://sites.google.com/view/orkneyshowresults}{\texttt{https://sites.google.com/view/orkneyshowresults}}. The $\TeX$ files used to create this document are available on GitHub here:\\ \href{https://github.com/robertmscarth/orkney-show-results}{\texttt{https://github.com/robertmscarth/orkney-show-results}}
	
	This is still very much a work-in-progress, and I'll add to it as and when I have the time.
	\\
\\
Robert Scarth, \\
%Birsay, \\
London, \\
\today

\newpage
	\section*{Copyright notice}
	This work is licensed under the Creative Commons Attribution-ShareAlike $4.0$ International License. To view a copy of this license, visit \newline \href{http://creativecommons.org/licenses/by-sa/4.0/}{\texttt{http://creativecommons.org/licenses/by-sa/4.0/}}.
	
	\subsection*{You are free to:}
	\begin{itemize}
		\item \textbf{Share} - copy and redistribute the material in any medium or format
		\item \textbf{Adapt} - remix, transform, and build upon the material for any purpose, even commercially.
	\end{itemize}
	The licensor cannot revoke these freedoms as long as you follow the license terms.\\ 
	\subsection*{Under the following terms:}
	\begin{itemize}
		\item \textbf{Attribution} - You must give appropriate credit, provide a link to the license, and indicate if changes were made. You may do so in any reasonable manner, but not in any way that suggests the licensor endorses you or your use.
		\item \textbf{Share Alike} - If you remix, transform, or build upon the material, you must distribute your contributions under the same license as the original.
	\end{itemize}
	\textbf{No additional restrictions} - You may not apply legal terms or technological measures that legally restrict others from doing anything the license permits.
	\subsection*{Notices:}
	You do not have to comply with the license for elements of the material in the public domain or where your use is permitted by an applicable exception or limitation.
	
	No warranties are given. The license may not give you all of the permissions necessary for your intended use. For example, other rights such as publicity, privacy, or moral rights may limit how you use the material.

\newpage

	\section*{Abbreviations}
	We use the following abbreviations for the various breeds:
	\subsection*{Beef Cattle}
	\begin{tabular}{cl}
		AA & Aberdeen Angus \\
		BB & Belgian Blue \\
		CH & Charolais \\
		LG & Luing \\
		LM & Limousin \\
		SH & Shorthorn\\
		SM & Simmental \\
	\end{tabular}

	\subsection*{Dairy Cattle}
	\begin{tabular}{cl}
		
	\end{tabular}
	
	\subsection*{Sheep}
	\begin{tabular}{cl}
		BL & Border Leicester \\
		CHV & Cheviot \\
		CHL & Charollais \\
		HB & Half-bred \\
		SF & Suffolk \\
		TX & Texel \\
	\end{tabular}

	\subsection*{Horses}
	\begin{tabular}{cl}
		
	\end{tabular}

	\subsection*{Goats}
	\begin{tabular}{cl}
		
	\end{tabular}

	\subsection*{Poultry}
	\begin{tabular}{cl}
		
	\end{tabular}

	\subsection*{Dogs}
	\begin{tabular}{cl}
		
	\end{tabular}

\newpage

	%use more of the paper for tables
	\addtolength{\textwidth}{3cm}
	\addtolength{\oddsidemargin}{-1.5cm}
	
	\begin{landscape}
		\chapter{Beef Cattle}
\section[Supreme Champion]{'Hope Show Supreme Champion $1970 - 2009$}
\small
\begin{longtable}{|c|p{5.2cm}|p{3cm}|p{3cm}|p{8cm}|}
\hline
	\textbf{Year} &
	\textbf{Exhibitor} &
	\textbf{Champion} &
	\textbf{Name} &
	\textbf{Notes} 
	\tabularnewline
\hline
\endhead
	$1978$ &
	\raggedright W. R. Baillie, Biggings, Toab\sindex[exhibitor]{Baillie, W. R., Biggings, Toab} &
	\raggedright 4yo bull &
	\raggedright Wincote Jasper\sindex[beef]{Wincote Jasper} &
	\raggedright By Bent Caribineer; bred by Lord Nelson of Stafford; bought at Northampton for 950gns; supreme champion in 1978 \& 1982.
	\tabularnewline
\hline
	$1979$ &
	\multicolumn{4}{c|}{No entries forward.}
	\tabularnewline
\hline
	$1980$ &
	\raggedright W. R. Baillie, Biggings, Toab\sindex[exhibitor]{Baillie, W. R., Biggings, Toab} &
	\raggedright 6yo bull &
	\raggedright Wincote Jasper\sindex[beef]{Wincote Jasper} &
	\multirow{3}{8cm}{By Bent Caribineer; bred by Lord Nelson of Stafford; bought at Northampton for 950gns; supreme champion in 1978 \& 1982.}
	\tabularnewline
\cline{1-4}
	$1981$ &
	\raggedright W. R. Baillie, Biggings, Toab\sindex[exhibitor]{Baillie, W. R., Biggings, Toab} &
	\raggedright 7yo bull &
	\raggedright Wincote Jasper\sindex[beef]{Wincote Jasper} &
	\tabularnewline
\cline{1-4}
	$1982$ &
	\raggedright W. R. Baillie, Biggings, Toab\sindex[exhibitor]{Baillie, W. R., Biggings, Toab} &
	\raggedright 8yo bull &
	\raggedright Wincote Jasper\sindex[beef]{Wincote Jasper} &
	\tabularnewline
\hline
	$1983$ &
	\raggedright W. R. Baillie, Biggings, Toab\sindex[exhibitor]{Baillie, W. R., Biggings, Toab} &
	\raggedright - &
	\raggedright - &
	\raggedright -
	\tabularnewline
\hline
	$1984$ &
	\raggedright W. R. Baillie, Biggings, Toab\sindex[exhibitor]{Baillie, W. R., Biggings, Toab} &
	\raggedright 10mo bull* &
	\raggedright Ballieston Universe\sindex[beef]{Ballieston Universe} &
	\raggedright By Wincote Jasper, out of Sway Ninette.
	\tabularnewline
\hline
	$1985$ &
	\raggedright G. T. \& S. Coghill, Muce, Birsay\sindex[exhibitor]{Coghill, G. T. \& S., Muce, Birsay} &
	\raggedright 3yo cow in calf &
	\raggedright Mornview Tamar\sindex[beef]{Mornview Tamar} &
	\raggedright By Tattenhall Hublot, out of a Mornview cow.
	\tabularnewline
\hline
	$1986$ &
	\raggedright W. G. Kirkpatrick \& Sons, Newhall, Stromness\sindex[exhibitor]{Kirkpatrick, W. G. \& Sons, Newhall, Stromness} &
	\raggedright 3yo bull &
	\raggedright Ballieston Universe\sindex[beef]{Ballieston Universe} &
	\raggedright By Wincote Jasper, out of Sway Ninette.
	\tabularnewline
\hline
	$1987$ &
	\raggedright G. T. \& S. Coghill, Muce, Birsay\sindex[exhibitor]{Coghill, G. T. \& S., Muce, Birsay} &
	\raggedright 8mo bull &
	\raggedright Hekra Challenger\sindex[beef]{Hekra Challenger} &
	\raggedright By Fairthwaite Magic, out of Hekra Vera.
	\tabularnewline
\hline
	$1988$ &
	\raggedright W. R. Baillie, Biggings, Toab\sindex[exhibitor]{Baillie, W. R., Biggings, Toab} &
	\raggedright 4yo cow* &
	\raggedright Ballieston Vaila\sindex[beef]{Ballieston Vaila} &
	\raggedright By Lunsdale Harvest, out of Ballieston Snowflake.
	\tabularnewline
\hline
	$1989$ &
	\raggedright J. M. Lennie \& Co, Nearhouse, Tankerness\sindex[exhibitor]{Lennie, J. M. \& Co, Nearhouse, Tankerness} &
	\raggedright 5yo cow in calf &
	\raggedright Little Gill Verity\sindex[beef]{Little Gill Verity} &
	\raggedright By Tatton Hall Hublot, out of Hole Moor Rebecca.
	\tabularnewline
\hline
	$1990$ &
	\raggedright D. W. Sinclair \& Co, Unigarth, Sandwick\sindex[exhibitor]{} &
	\raggedright 2yo bull &
	\raggedright Royston Democrat\sindex[beef]{Royston Democrat} &
	\raggedright By Glenleary Attenhall.
	\tabularnewline
\hline
	$1991$ &
	\raggedright J. M. Lennie \& Co, Nearhouse, Tankerness\sindex[exhibitor]{Lennie, J. M. \& Co, Nearhouse, Tankerness} &
	\raggedright 7yo cow &
	\raggedright Balthayock Violet\sindex[beef]{Balthayock Violet} &
	\raggedright By Balthayock Oliver.
	\tabularnewline
\hline
	$1992$ &
	\raggedright J. M. Lennie \& Co, Nearhouse, Tankerness\sindex[exhibitor]{Lennie, J. M. \& Co, Nearhouse, Tankerness} &
	\raggedright 9yo cow in milk &
	\raggedright Little Gill Verity\sindex[beef]{Little Gill Verity} &
	\raggedright By Tattenhall Hiblot, out of Holehouse Rebecca.
	\tabularnewline
\hline
	$1993$ &
	\raggedright W. R. Baillie, Biggings, Toab\sindex[exhibitor]{Baillie, W. R., Biggings, Toab} &
	\raggedright 18mo heifer* &
	\raggedright Ballieston Helga\sindex[beef]{Ballieston Helga} &
	\raggedright By Ballieston Donald, out of Ballieston Catherine.
	\tabularnewline
\hline
	$1994$ &
	\raggedright W. R. Baillie, Biggings, Toab\sindex[exhibitor]{Baillie, W. R., Biggings, Toab} &
	\raggedright 18mo bull* &
	\raggedright Ballieston Howard\sindex[beef]{Ballieston Howard} &
	\raggedright By Cockrington Baron, out of Ballieston Frances.
	\tabularnewline
\hline
	$1995$ &
	\raggedright W. R. Baillie, Biggings, Toab\sindex[exhibitor]{Baillie, W. R., Biggings, Toab} &
	\raggedright 15mo heifer* &
	\raggedright Ballieston Jacqueline\sindex[beef]{Ballieston Jacqueline} &
	\multirow{2}{8cm}{By Fleets Rascal, out of Ballieston Donna.}
	\tabularnewline
\cline{1-4}
	$1996$ &
	\raggedright W. R. Baillie, Biggings, Toab\sindex[exhibitor]{Baillie, W. R., Biggings, Toab} &
	\raggedright 20mo heifer* &
	\raggedright Ballieston Jacqueline\sindex[beef]{Ballieston Jacqueline} &
	\tabularnewline
\hline
	$1997$ &
	\raggedright J. A. Scott, Howe, Harray\sindex[exhibitor]{Scott, J. A., Howe, Harray} &
	\raggedright 6yo bull &
	\raggedright Seawell Goutte\sindex[beef]{Seawell Goutte} &
	\raggedright By Seawell Caribbean; bought at Perth as a 13mo.
	\tabularnewline
\hline
	$1998$ &
	\raggedright G. T. \& S. Coghill, Muce, Birsay\sindex[exhibitor]{Coghill, G. T. \& S., Muce, Birsay} &
	\raggedright 5yo cow in milk &
	\raggedright New Moan Iris\sindex[beef]{New Moan Iris} &
	\raggedright By Balthayoak Dalesman, out of New Moan Emma; bought from Jim Isbister, New Moan, Harray, in October 1997.
	\tabularnewline
\hline
	$1999$ &
	\raggedright Garson Farms, Sandwick\sindex[exhibitor]{Garson Farms, Sandwick} &
	\raggedright 6yo cow* &
	\raggedright Garson Isla\sindex[beef]{Garson Isla} &
	\raggedright By Brampton Consort, out of Gredington Caniche.
	\tabularnewline
\hline
	$2000$ &
	\raggedright W. R. Baillie, Biggings, Toab\sindex[exhibitor]{Baillie, W. R., Biggings, Toab} &
	\raggedright 7yo dry cow* &
	\raggedright Ballieston Ingrid\sindex[beef]{Ballieston Ingrid} &
	\raggedright By Maerdy Victorious, out of Ballieston Ursula.
	\tabularnewline
\hline
	$2001$ &
	\multicolumn{4}{c|}{No Show due to Foot \& Mouth outbreak.}
	\tabularnewline
\hline
	$2002$ &
	\raggedright L. Spence, Spengar, Sandwick\sindex[exhibitor]{Spence, L., Spengar, Sandwick} &
	\raggedright 16mo bull* &
	\raggedright Spengar Sambhar\sindex[beef]{Spengar Sambhar} &
	\raggedright By Brampton Four Star, out of Mortimer's Lauren.
	\tabularnewline
\hline
	$2003$ &
	\raggedright Garson Farms, Sandwick\sindex[exhibitor]{Garson Farms, Sandwick} &
	\raggedright 1yo heifer &
	\raggedright Airdlin Treasure\sindex[beef]{Airdlin Treasure} &
	\raggedright By Airdlin Ploughman, out of Maerdy Limpet; bought at Perth February 2003.
	\tabularnewline
\hline
	$2004$ &
	\raggedright W. R. Baillie, Biggings, Toab\sindex[exhibitor]{Baillie, W. R., Biggings, Toab} &
	\raggedright 17mo heifer* &
	\raggedright Baillieston Topsy\sindex[beef]{Baillieston Topsy} &
	\raggedright By Moyness President, out of Ballieston Maiden.
	\tabularnewline
\hline
	$2005$ &
	\raggedright Messrs Davidson, Skaill, Sandwick\sindex[exhibitor]{Davidson, Messrs, Skaill, Sandwick} &
	\raggedright 16mo bull* &
	\raggedright Skaill Victor\sindex[beef]{Skaill Victor} &
	\raggedright By Offenbach, out of Rosette; supreme champion 2005.
	\tabularnewline
\hline
	$2006$ &
	\raggedright Messrs Davidson, Skaill, Sandwick\sindex[exhibitor]{Davidson, Messrs, Skaill, Sandwick} &
	\raggedright 16mo heifer* &
	\raggedright Skaill Abbie\sindex[beef]{Skaill Abbie} &
	\raggedright By Nectragerc, out of Perche; supreme champion, 2006; supreme champion at 2006 Dounby Show.
	\tabularnewline
\hline
	$2007$ &
	\multicolumn{4}{c|}{No Show due to Foot \& Mouth outbreak.}
	\tabularnewline
\hline
	$2008$ &
	\raggedright W. R. Baillie, Biggings, Toab\sindex[exhibitor]{Baillie, W. R., Biggins, Toab} &
	\raggedright 2yo heifer* &
	\raggedright Baillieston Anthea\sindex[beef]{Baillieston Anthea} &
	\raggedright By Mowbraypark Umpire, out of Baillieston Umbreta; supreme champion.
	\tabularnewline
\hline
	$2009$ &
	\raggedright W. R. Baillie, Biggings, Toab\sindex[exhibitor]{Baillie, W. R., Biggins, Toab} &
	\raggedright 5yo cow-in-milk* &
	\raggedright Baillieston Vera\sindex[beef]{Baillieston Anthea} &
	\raggedright By Moyness President, out of Baillieston Ola; supreme champion.
	\tabularnewline
\hline
	$2010$ &
	\raggedright J. M. Lennie \& Co, Nearhouse, Tankerness\sindex[exhibitor]{Lennie, J. M. \& Co, Nearhouse, Tankerness} &
	\raggedright 4yo bull* &
	\raggedright Nearhouse Bobby &
	\raggedright By Fremch bull Oscar, out of Brymire Ola.
	\tabularnewline
\hline
	$2011$ &
	\raggedright  &
	\raggedright  &
	\raggedright  &
	\raggedright 
	\tabularnewline
\hline
\end{longtable}

\normalsize

		\chapter{Beef Cattle}
\section[Supreme Champion]{'Hope Show Supreme Champion $1970 - 2009$}
\small
\begin{longtable}{|c|p{5.2cm}|p{3cm}|p{3cm}|p{8cm}|}
\hline
	\textbf{Year} &
	\textbf{Exhibitor} &
	\textbf{Champion} &
	\textbf{Name} &
	\textbf{Notes} 
	\tabularnewline
\hline
\endhead
	$1978$ &
	\raggedright W. R. Baillie, Biggings, Toab\sindex[exhibitor]{Baillie, W. R., Biggings, Toab} &
	\raggedright 4yo bull &
	\raggedright Wincote Jasper\sindex[beef]{Wincote Jasper} &
	\raggedright By Bent Caribineer; bred by Lord Nelson of Stafford; bought at Northampton for 950gns; supreme champion in 1978 \& 1982.
	\tabularnewline
\hline
	$1979$ &
	\multicolumn{4}{c|}{No entries forward.}
	\tabularnewline
\hline
	$1980$ &
	\raggedright W. R. Baillie, Biggings, Toab\sindex[exhibitor]{Baillie, W. R., Biggings, Toab} &
	\raggedright 6yo bull &
	\raggedright Wincote Jasper\sindex[beef]{Wincote Jasper} &
	\multirow{3}{8cm}{By Bent Caribineer; bred by Lord Nelson of Stafford; bought at Northampton for 950gns; supreme champion in 1978 \& 1982.}
	\tabularnewline
\cline{1-4}
	$1981$ &
	\raggedright W. R. Baillie, Biggings, Toab\sindex[exhibitor]{Baillie, W. R., Biggings, Toab} &
	\raggedright 7yo bull &
	\raggedright Wincote Jasper\sindex[beef]{Wincote Jasper} &
	\tabularnewline
\cline{1-4}
	$1982$ &
	\raggedright W. R. Baillie, Biggings, Toab\sindex[exhibitor]{Baillie, W. R., Biggings, Toab} &
	\raggedright 8yo bull &
	\raggedright Wincote Jasper\sindex[beef]{Wincote Jasper} &
	\tabularnewline
\hline
	$1983$ &
	\raggedright W. R. Baillie, Biggings, Toab\sindex[exhibitor]{Baillie, W. R., Biggings, Toab} &
	\raggedright - &
	\raggedright - &
	\raggedright -
	\tabularnewline
\hline
	$1984$ &
	\raggedright W. R. Baillie, Biggings, Toab\sindex[exhibitor]{Baillie, W. R., Biggings, Toab} &
	\raggedright 10mo bull* &
	\raggedright Ballieston Universe\sindex[beef]{Ballieston Universe} &
	\raggedright By Wincote Jasper, out of Sway Ninette.
	\tabularnewline
\hline
	$1985$ &
	\raggedright G. T. \& S. Coghill, Muce, Birsay\sindex[exhibitor]{Coghill, G. T. \& S., Muce, Birsay} &
	\raggedright 3yo cow in calf &
	\raggedright Mornview Tamar\sindex[beef]{Mornview Tamar} &
	\raggedright By Tattenhall Hublot, out of a Mornview cow.
	\tabularnewline
\hline
	$1986$ &
	\raggedright W. G. Kirkpatrick \& Sons, Newhall, Stromness\sindex[exhibitor]{Kirkpatrick, W. G. \& Sons, Newhall, Stromness} &
	\raggedright 3yo bull &
	\raggedright Ballieston Universe\sindex[beef]{Ballieston Universe} &
	\raggedright By Wincote Jasper, out of Sway Ninette.
	\tabularnewline
\hline
	$1987$ &
	\raggedright G. T. \& S. Coghill, Muce, Birsay\sindex[exhibitor]{Coghill, G. T. \& S., Muce, Birsay} &
	\raggedright 8mo bull &
	\raggedright Hekra Challenger\sindex[beef]{Hekra Challenger} &
	\raggedright By Fairthwaite Magic, out of Hekra Vera.
	\tabularnewline
\hline
	$1988$ &
	\raggedright W. R. Baillie, Biggings, Toab\sindex[exhibitor]{Baillie, W. R., Biggings, Toab} &
	\raggedright 4yo cow* &
	\raggedright Ballieston Vaila\sindex[beef]{Ballieston Vaila} &
	\raggedright By Lunsdale Harvest, out of Ballieston Snowflake.
	\tabularnewline
\hline
	$1989$ &
	\raggedright J. M. Lennie \& Co, Nearhouse, Tankerness\sindex[exhibitor]{Lennie, J. M. \& Co, Nearhouse, Tankerness} &
	\raggedright 5yo cow in calf &
	\raggedright Little Gill Verity\sindex[beef]{Little Gill Verity} &
	\raggedright By Tatton Hall Hublot, out of Hole Moor Rebecca.
	\tabularnewline
\hline
	$1990$ &
	\raggedright D. W. Sinclair \& Co, Unigarth, Sandwick\sindex[exhibitor]{} &
	\raggedright 2yo bull &
	\raggedright Royston Democrat\sindex[beef]{Royston Democrat} &
	\raggedright By Glenleary Attenhall.
	\tabularnewline
\hline
	$1991$ &
	\raggedright J. M. Lennie \& Co, Nearhouse, Tankerness\sindex[exhibitor]{Lennie, J. M. \& Co, Nearhouse, Tankerness} &
	\raggedright 7yo cow &
	\raggedright Balthayock Violet\sindex[beef]{Balthayock Violet} &
	\raggedright By Balthayock Oliver.
	\tabularnewline
\hline
	$1992$ &
	\raggedright J. M. Lennie \& Co, Nearhouse, Tankerness\sindex[exhibitor]{Lennie, J. M. \& Co, Nearhouse, Tankerness} &
	\raggedright 9yo cow in milk &
	\raggedright Little Gill Verity\sindex[beef]{Little Gill Verity} &
	\raggedright By Tattenhall Hiblot, out of Holehouse Rebecca.
	\tabularnewline
\hline
	$1993$ &
	\raggedright W. R. Baillie, Biggings, Toab\sindex[exhibitor]{Baillie, W. R., Biggings, Toab} &
	\raggedright 18mo heifer* &
	\raggedright Ballieston Helga\sindex[beef]{Ballieston Helga} &
	\raggedright By Ballieston Donald, out of Ballieston Catherine.
	\tabularnewline
\hline
	$1994$ &
	\raggedright W. R. Baillie, Biggings, Toab\sindex[exhibitor]{Baillie, W. R., Biggings, Toab} &
	\raggedright 18mo bull* &
	\raggedright Ballieston Howard\sindex[beef]{Ballieston Howard} &
	\raggedright By Cockrington Baron, out of Ballieston Frances.
	\tabularnewline
\hline
	$1995$ &
	\raggedright W. R. Baillie, Biggings, Toab\sindex[exhibitor]{Baillie, W. R., Biggings, Toab} &
	\raggedright 15mo heifer* &
	\raggedright Ballieston Jacqueline\sindex[beef]{Ballieston Jacqueline} &
	\multirow{2}{8cm}{By Fleets Rascal, out of Ballieston Donna.}
	\tabularnewline
\cline{1-4}
	$1996$ &
	\raggedright W. R. Baillie, Biggings, Toab\sindex[exhibitor]{Baillie, W. R., Biggings, Toab} &
	\raggedright 20mo heifer* &
	\raggedright Ballieston Jacqueline\sindex[beef]{Ballieston Jacqueline} &
	\tabularnewline
\hline
	$1997$ &
	\raggedright J. A. Scott, Howe, Harray\sindex[exhibitor]{Scott, J. A., Howe, Harray} &
	\raggedright 6yo bull &
	\raggedright Seawell Goutte\sindex[beef]{Seawell Goutte} &
	\raggedright By Seawell Caribbean; bought at Perth as a 13mo.
	\tabularnewline
\hline
	$1998$ &
	\raggedright G. T. \& S. Coghill, Muce, Birsay\sindex[exhibitor]{Coghill, G. T. \& S., Muce, Birsay} &
	\raggedright 5yo cow in milk &
	\raggedright New Moan Iris\sindex[beef]{New Moan Iris} &
	\raggedright By Balthayoak Dalesman, out of New Moan Emma; bought from Jim Isbister, New Moan, Harray, in October 1997.
	\tabularnewline
\hline
	$1999$ &
	\raggedright Garson Farms, Sandwick\sindex[exhibitor]{Garson Farms, Sandwick} &
	\raggedright 6yo cow* &
	\raggedright Garson Isla\sindex[beef]{Garson Isla} &
	\raggedright By Brampton Consort, out of Gredington Caniche.
	\tabularnewline
\hline
	$2000$ &
	\raggedright W. R. Baillie, Biggings, Toab\sindex[exhibitor]{Baillie, W. R., Biggings, Toab} &
	\raggedright 7yo dry cow* &
	\raggedright Ballieston Ingrid\sindex[beef]{Ballieston Ingrid} &
	\raggedright By Maerdy Victorious, out of Ballieston Ursula.
	\tabularnewline
\hline
	$2001$ &
	\multicolumn{4}{c|}{No Show due to Foot \& Mouth outbreak.}
	\tabularnewline
\hline
	$2002$ &
	\raggedright L. Spence, Spengar, Sandwick\sindex[exhibitor]{Spence, L., Spengar, Sandwick} &
	\raggedright 16mo bull* &
	\raggedright Spengar Sambhar\sindex[beef]{Spengar Sambhar} &
	\raggedright By Brampton Four Star, out of Mortimer's Lauren.
	\tabularnewline
\hline
	$2003$ &
	\raggedright Garson Farms, Sandwick\sindex[exhibitor]{Garson Farms, Sandwick} &
	\raggedright 1yo heifer &
	\raggedright Airdlin Treasure\sindex[beef]{Airdlin Treasure} &
	\raggedright By Airdlin Ploughman, out of Maerdy Limpet; bought at Perth February 2003.
	\tabularnewline
\hline
	$2004$ &
	\raggedright W. R. Baillie, Biggings, Toab\sindex[exhibitor]{Baillie, W. R., Biggings, Toab} &
	\raggedright 17mo heifer* &
	\raggedright Baillieston Topsy\sindex[beef]{Baillieston Topsy} &
	\raggedright By Moyness President, out of Ballieston Maiden.
	\tabularnewline
\hline
	$2005$ &
	\raggedright Messrs Davidson, Skaill, Sandwick\sindex[exhibitor]{Davidson, Messrs, Skaill, Sandwick} &
	\raggedright 16mo bull* &
	\raggedright Skaill Victor\sindex[beef]{Skaill Victor} &
	\raggedright By Offenbach, out of Rosette; supreme champion 2005.
	\tabularnewline
\hline
	$2006$ &
	\raggedright Messrs Davidson, Skaill, Sandwick\sindex[exhibitor]{Davidson, Messrs, Skaill, Sandwick} &
	\raggedright 16mo heifer* &
	\raggedright Skaill Abbie\sindex[beef]{Skaill Abbie} &
	\raggedright By Nectragerc, out of Perche; supreme champion, 2006; supreme champion at 2006 Dounby Show.
	\tabularnewline
\hline
	$2007$ &
	\multicolumn{4}{c|}{No Show due to Foot \& Mouth outbreak.}
	\tabularnewline
\hline
	$2008$ &
	\raggedright W. R. Baillie, Biggings, Toab\sindex[exhibitor]{Baillie, W. R., Biggins, Toab} &
	\raggedright 2yo heifer* &
	\raggedright Baillieston Anthea\sindex[beef]{Baillieston Anthea} &
	\raggedright By Mowbraypark Umpire, out of Baillieston Umbreta; supreme champion.
	\tabularnewline
\hline
	$2009$ &
	\raggedright W. R. Baillie, Biggings, Toab\sindex[exhibitor]{Baillie, W. R., Biggins, Toab} &
	\raggedright 5yo cow-in-milk* &
	\raggedright Baillieston Vera\sindex[beef]{Baillieston Anthea} &
	\raggedright By Moyness President, out of Baillieston Ola; supreme champion.
	\tabularnewline
\hline
	$2010$ &
	\raggedright J. M. Lennie \& Co, Nearhouse, Tankerness\sindex[exhibitor]{Lennie, J. M. \& Co, Nearhouse, Tankerness} &
	\raggedright 4yo bull* &
	\raggedright Nearhouse Bobby &
	\raggedright By Fremch bull Oscar, out of Brymire Ola.
	\tabularnewline
\hline
	$2011$ &
	\raggedright  &
	\raggedright  &
	\raggedright  &
	\raggedright 
	\tabularnewline
\hline
\end{longtable}

\normalsize

		\chapter{Beef Cattle}
\section[Supreme Champion]{'Hope Show Supreme Champion $1970 - 2009$}
\small
\begin{longtable}{|c|p{5.2cm}|p{3cm}|p{3cm}|p{8cm}|}
\hline
	\textbf{Year} &
	\textbf{Exhibitor} &
	\textbf{Champion} &
	\textbf{Name} &
	\textbf{Notes} 
	\tabularnewline
\hline
\endhead
	$1978$ &
	\raggedright W. R. Baillie, Biggings, Toab\sindex[exhibitor]{Baillie, W. R., Biggings, Toab} &
	\raggedright 4yo bull &
	\raggedright Wincote Jasper\sindex[beef]{Wincote Jasper} &
	\raggedright By Bent Caribineer; bred by Lord Nelson of Stafford; bought at Northampton for 950gns; supreme champion in 1978 \& 1982.
	\tabularnewline
\hline
	$1979$ &
	\multicolumn{4}{c|}{No entries forward.}
	\tabularnewline
\hline
	$1980$ &
	\raggedright W. R. Baillie, Biggings, Toab\sindex[exhibitor]{Baillie, W. R., Biggings, Toab} &
	\raggedright 6yo bull &
	\raggedright Wincote Jasper\sindex[beef]{Wincote Jasper} &
	\multirow{3}{8cm}{By Bent Caribineer; bred by Lord Nelson of Stafford; bought at Northampton for 950gns; supreme champion in 1978 \& 1982.}
	\tabularnewline
\cline{1-4}
	$1981$ &
	\raggedright W. R. Baillie, Biggings, Toab\sindex[exhibitor]{Baillie, W. R., Biggings, Toab} &
	\raggedright 7yo bull &
	\raggedright Wincote Jasper\sindex[beef]{Wincote Jasper} &
	\tabularnewline
\cline{1-4}
	$1982$ &
	\raggedright W. R. Baillie, Biggings, Toab\sindex[exhibitor]{Baillie, W. R., Biggings, Toab} &
	\raggedright 8yo bull &
	\raggedright Wincote Jasper\sindex[beef]{Wincote Jasper} &
	\tabularnewline
\hline
	$1983$ &
	\raggedright W. R. Baillie, Biggings, Toab\sindex[exhibitor]{Baillie, W. R., Biggings, Toab} &
	\raggedright - &
	\raggedright - &
	\raggedright -
	\tabularnewline
\hline
	$1984$ &
	\raggedright W. R. Baillie, Biggings, Toab\sindex[exhibitor]{Baillie, W. R., Biggings, Toab} &
	\raggedright 10mo bull* &
	\raggedright Ballieston Universe\sindex[beef]{Ballieston Universe} &
	\raggedright By Wincote Jasper, out of Sway Ninette.
	\tabularnewline
\hline
	$1985$ &
	\raggedright G. T. \& S. Coghill, Muce, Birsay\sindex[exhibitor]{Coghill, G. T. \& S., Muce, Birsay} &
	\raggedright 3yo cow in calf &
	\raggedright Mornview Tamar\sindex[beef]{Mornview Tamar} &
	\raggedright By Tattenhall Hublot, out of a Mornview cow.
	\tabularnewline
\hline
	$1986$ &
	\raggedright W. G. Kirkpatrick \& Sons, Newhall, Stromness\sindex[exhibitor]{Kirkpatrick, W. G. \& Sons, Newhall, Stromness} &
	\raggedright 3yo bull &
	\raggedright Ballieston Universe\sindex[beef]{Ballieston Universe} &
	\raggedright By Wincote Jasper, out of Sway Ninette.
	\tabularnewline
\hline
	$1987$ &
	\raggedright G. T. \& S. Coghill, Muce, Birsay\sindex[exhibitor]{Coghill, G. T. \& S., Muce, Birsay} &
	\raggedright 8mo bull &
	\raggedright Hekra Challenger\sindex[beef]{Hekra Challenger} &
	\raggedright By Fairthwaite Magic, out of Hekra Vera.
	\tabularnewline
\hline
	$1988$ &
	\raggedright W. R. Baillie, Biggings, Toab\sindex[exhibitor]{Baillie, W. R., Biggings, Toab} &
	\raggedright 4yo cow* &
	\raggedright Ballieston Vaila\sindex[beef]{Ballieston Vaila} &
	\raggedright By Lunsdale Harvest, out of Ballieston Snowflake.
	\tabularnewline
\hline
	$1989$ &
	\raggedright J. M. Lennie \& Co, Nearhouse, Tankerness\sindex[exhibitor]{Lennie, J. M. \& Co, Nearhouse, Tankerness} &
	\raggedright 5yo cow in calf &
	\raggedright Little Gill Verity\sindex[beef]{Little Gill Verity} &
	\raggedright By Tatton Hall Hublot, out of Hole Moor Rebecca.
	\tabularnewline
\hline
	$1990$ &
	\raggedright D. W. Sinclair \& Co, Unigarth, Sandwick\sindex[exhibitor]{} &
	\raggedright 2yo bull &
	\raggedright Royston Democrat\sindex[beef]{Royston Democrat} &
	\raggedright By Glenleary Attenhall.
	\tabularnewline
\hline
	$1991$ &
	\raggedright J. M. Lennie \& Co, Nearhouse, Tankerness\sindex[exhibitor]{Lennie, J. M. \& Co, Nearhouse, Tankerness} &
	\raggedright 7yo cow &
	\raggedright Balthayock Violet\sindex[beef]{Balthayock Violet} &
	\raggedright By Balthayock Oliver.
	\tabularnewline
\hline
	$1992$ &
	\raggedright J. M. Lennie \& Co, Nearhouse, Tankerness\sindex[exhibitor]{Lennie, J. M. \& Co, Nearhouse, Tankerness} &
	\raggedright 9yo cow in milk &
	\raggedright Little Gill Verity\sindex[beef]{Little Gill Verity} &
	\raggedright By Tattenhall Hiblot, out of Holehouse Rebecca.
	\tabularnewline
\hline
	$1993$ &
	\raggedright W. R. Baillie, Biggings, Toab\sindex[exhibitor]{Baillie, W. R., Biggings, Toab} &
	\raggedright 18mo heifer* &
	\raggedright Ballieston Helga\sindex[beef]{Ballieston Helga} &
	\raggedright By Ballieston Donald, out of Ballieston Catherine.
	\tabularnewline
\hline
	$1994$ &
	\raggedright W. R. Baillie, Biggings, Toab\sindex[exhibitor]{Baillie, W. R., Biggings, Toab} &
	\raggedright 18mo bull* &
	\raggedright Ballieston Howard\sindex[beef]{Ballieston Howard} &
	\raggedright By Cockrington Baron, out of Ballieston Frances.
	\tabularnewline
\hline
	$1995$ &
	\raggedright W. R. Baillie, Biggings, Toab\sindex[exhibitor]{Baillie, W. R., Biggings, Toab} &
	\raggedright 15mo heifer* &
	\raggedright Ballieston Jacqueline\sindex[beef]{Ballieston Jacqueline} &
	\multirow{2}{8cm}{By Fleets Rascal, out of Ballieston Donna.}
	\tabularnewline
\cline{1-4}
	$1996$ &
	\raggedright W. R. Baillie, Biggings, Toab\sindex[exhibitor]{Baillie, W. R., Biggings, Toab} &
	\raggedright 20mo heifer* &
	\raggedright Ballieston Jacqueline\sindex[beef]{Ballieston Jacqueline} &
	\tabularnewline
\hline
	$1997$ &
	\raggedright J. A. Scott, Howe, Harray\sindex[exhibitor]{Scott, J. A., Howe, Harray} &
	\raggedright 6yo bull &
	\raggedright Seawell Goutte\sindex[beef]{Seawell Goutte} &
	\raggedright By Seawell Caribbean; bought at Perth as a 13mo.
	\tabularnewline
\hline
	$1998$ &
	\raggedright G. T. \& S. Coghill, Muce, Birsay\sindex[exhibitor]{Coghill, G. T. \& S., Muce, Birsay} &
	\raggedright 5yo cow in milk &
	\raggedright New Moan Iris\sindex[beef]{New Moan Iris} &
	\raggedright By Balthayoak Dalesman, out of New Moan Emma; bought from Jim Isbister, New Moan, Harray, in October 1997.
	\tabularnewline
\hline
	$1999$ &
	\raggedright Garson Farms, Sandwick\sindex[exhibitor]{Garson Farms, Sandwick} &
	\raggedright 6yo cow* &
	\raggedright Garson Isla\sindex[beef]{Garson Isla} &
	\raggedright By Brampton Consort, out of Gredington Caniche.
	\tabularnewline
\hline
	$2000$ &
	\raggedright W. R. Baillie, Biggings, Toab\sindex[exhibitor]{Baillie, W. R., Biggings, Toab} &
	\raggedright 7yo dry cow* &
	\raggedright Ballieston Ingrid\sindex[beef]{Ballieston Ingrid} &
	\raggedright By Maerdy Victorious, out of Ballieston Ursula.
	\tabularnewline
\hline
	$2001$ &
	\multicolumn{4}{c|}{No Show due to Foot \& Mouth outbreak.}
	\tabularnewline
\hline
	$2002$ &
	\raggedright L. Spence, Spengar, Sandwick\sindex[exhibitor]{Spence, L., Spengar, Sandwick} &
	\raggedright 16mo bull* &
	\raggedright Spengar Sambhar\sindex[beef]{Spengar Sambhar} &
	\raggedright By Brampton Four Star, out of Mortimer's Lauren.
	\tabularnewline
\hline
	$2003$ &
	\raggedright Garson Farms, Sandwick\sindex[exhibitor]{Garson Farms, Sandwick} &
	\raggedright 1yo heifer &
	\raggedright Airdlin Treasure\sindex[beef]{Airdlin Treasure} &
	\raggedright By Airdlin Ploughman, out of Maerdy Limpet; bought at Perth February 2003.
	\tabularnewline
\hline
	$2004$ &
	\raggedright W. R. Baillie, Biggings, Toab\sindex[exhibitor]{Baillie, W. R., Biggings, Toab} &
	\raggedright 17mo heifer* &
	\raggedright Baillieston Topsy\sindex[beef]{Baillieston Topsy} &
	\raggedright By Moyness President, out of Ballieston Maiden.
	\tabularnewline
\hline
	$2005$ &
	\raggedright Messrs Davidson, Skaill, Sandwick\sindex[exhibitor]{Davidson, Messrs, Skaill, Sandwick} &
	\raggedright 16mo bull* &
	\raggedright Skaill Victor\sindex[beef]{Skaill Victor} &
	\raggedright By Offenbach, out of Rosette; supreme champion 2005.
	\tabularnewline
\hline
	$2006$ &
	\raggedright Messrs Davidson, Skaill, Sandwick\sindex[exhibitor]{Davidson, Messrs, Skaill, Sandwick} &
	\raggedright 16mo heifer* &
	\raggedright Skaill Abbie\sindex[beef]{Skaill Abbie} &
	\raggedright By Nectragerc, out of Perche; supreme champion, 2006; supreme champion at 2006 Dounby Show.
	\tabularnewline
\hline
	$2007$ &
	\multicolumn{4}{c|}{No Show due to Foot \& Mouth outbreak.}
	\tabularnewline
\hline
	$2008$ &
	\raggedright W. R. Baillie, Biggings, Toab\sindex[exhibitor]{Baillie, W. R., Biggins, Toab} &
	\raggedright 2yo heifer* &
	\raggedright Baillieston Anthea\sindex[beef]{Baillieston Anthea} &
	\raggedright By Mowbraypark Umpire, out of Baillieston Umbreta; supreme champion.
	\tabularnewline
\hline
	$2009$ &
	\raggedright W. R. Baillie, Biggings, Toab\sindex[exhibitor]{Baillie, W. R., Biggins, Toab} &
	\raggedright 5yo cow-in-milk* &
	\raggedright Baillieston Vera\sindex[beef]{Baillieston Anthea} &
	\raggedright By Moyness President, out of Baillieston Ola; supreme champion.
	\tabularnewline
\hline
	$2010$ &
	\raggedright J. M. Lennie \& Co, Nearhouse, Tankerness\sindex[exhibitor]{Lennie, J. M. \& Co, Nearhouse, Tankerness} &
	\raggedright 4yo bull* &
	\raggedright Nearhouse Bobby &
	\raggedright By Fremch bull Oscar, out of Brymire Ola.
	\tabularnewline
\hline
	$2011$ &
	\raggedright  &
	\raggedright  &
	\raggedright  &
	\raggedright 
	\tabularnewline
\hline
\end{longtable}

\normalsize

		\chapter{Beef Cattle}
\section[Supreme Champion]{'Hope Show Supreme Champion $1970 - 2009$}
\small
\begin{longtable}{|c|p{5.2cm}|p{3cm}|p{3cm}|p{8cm}|}
\hline
	\textbf{Year} &
	\textbf{Exhibitor} &
	\textbf{Champion} &
	\textbf{Name} &
	\textbf{Notes} 
	\tabularnewline
\hline
\endhead
	$1978$ &
	\raggedright W. R. Baillie, Biggings, Toab\sindex[exhibitor]{Baillie, W. R., Biggings, Toab} &
	\raggedright 4yo bull &
	\raggedright Wincote Jasper\sindex[beef]{Wincote Jasper} &
	\raggedright By Bent Caribineer; bred by Lord Nelson of Stafford; bought at Northampton for 950gns; supreme champion in 1978 \& 1982.
	\tabularnewline
\hline
	$1979$ &
	\multicolumn{4}{c|}{No entries forward.}
	\tabularnewline
\hline
	$1980$ &
	\raggedright W. R. Baillie, Biggings, Toab\sindex[exhibitor]{Baillie, W. R., Biggings, Toab} &
	\raggedright 6yo bull &
	\raggedright Wincote Jasper\sindex[beef]{Wincote Jasper} &
	\multirow{3}{8cm}{By Bent Caribineer; bred by Lord Nelson of Stafford; bought at Northampton for 950gns; supreme champion in 1978 \& 1982.}
	\tabularnewline
\cline{1-4}
	$1981$ &
	\raggedright W. R. Baillie, Biggings, Toab\sindex[exhibitor]{Baillie, W. R., Biggings, Toab} &
	\raggedright 7yo bull &
	\raggedright Wincote Jasper\sindex[beef]{Wincote Jasper} &
	\tabularnewline
\cline{1-4}
	$1982$ &
	\raggedright W. R. Baillie, Biggings, Toab\sindex[exhibitor]{Baillie, W. R., Biggings, Toab} &
	\raggedright 8yo bull &
	\raggedright Wincote Jasper\sindex[beef]{Wincote Jasper} &
	\tabularnewline
\hline
	$1983$ &
	\raggedright W. R. Baillie, Biggings, Toab\sindex[exhibitor]{Baillie, W. R., Biggings, Toab} &
	\raggedright - &
	\raggedright - &
	\raggedright -
	\tabularnewline
\hline
	$1984$ &
	\raggedright W. R. Baillie, Biggings, Toab\sindex[exhibitor]{Baillie, W. R., Biggings, Toab} &
	\raggedright 10mo bull* &
	\raggedright Ballieston Universe\sindex[beef]{Ballieston Universe} &
	\raggedright By Wincote Jasper, out of Sway Ninette.
	\tabularnewline
\hline
	$1985$ &
	\raggedright G. T. \& S. Coghill, Muce, Birsay\sindex[exhibitor]{Coghill, G. T. \& S., Muce, Birsay} &
	\raggedright 3yo cow in calf &
	\raggedright Mornview Tamar\sindex[beef]{Mornview Tamar} &
	\raggedright By Tattenhall Hublot, out of a Mornview cow.
	\tabularnewline
\hline
	$1986$ &
	\raggedright W. G. Kirkpatrick \& Sons, Newhall, Stromness\sindex[exhibitor]{Kirkpatrick, W. G. \& Sons, Newhall, Stromness} &
	\raggedright 3yo bull &
	\raggedright Ballieston Universe\sindex[beef]{Ballieston Universe} &
	\raggedright By Wincote Jasper, out of Sway Ninette.
	\tabularnewline
\hline
	$1987$ &
	\raggedright G. T. \& S. Coghill, Muce, Birsay\sindex[exhibitor]{Coghill, G. T. \& S., Muce, Birsay} &
	\raggedright 8mo bull &
	\raggedright Hekra Challenger\sindex[beef]{Hekra Challenger} &
	\raggedright By Fairthwaite Magic, out of Hekra Vera.
	\tabularnewline
\hline
	$1988$ &
	\raggedright W. R. Baillie, Biggings, Toab\sindex[exhibitor]{Baillie, W. R., Biggings, Toab} &
	\raggedright 4yo cow* &
	\raggedright Ballieston Vaila\sindex[beef]{Ballieston Vaila} &
	\raggedright By Lunsdale Harvest, out of Ballieston Snowflake.
	\tabularnewline
\hline
	$1989$ &
	\raggedright J. M. Lennie \& Co, Nearhouse, Tankerness\sindex[exhibitor]{Lennie, J. M. \& Co, Nearhouse, Tankerness} &
	\raggedright 5yo cow in calf &
	\raggedright Little Gill Verity\sindex[beef]{Little Gill Verity} &
	\raggedright By Tatton Hall Hublot, out of Hole Moor Rebecca.
	\tabularnewline
\hline
	$1990$ &
	\raggedright D. W. Sinclair \& Co, Unigarth, Sandwick\sindex[exhibitor]{} &
	\raggedright 2yo bull &
	\raggedright Royston Democrat\sindex[beef]{Royston Democrat} &
	\raggedright By Glenleary Attenhall.
	\tabularnewline
\hline
	$1991$ &
	\raggedright J. M. Lennie \& Co, Nearhouse, Tankerness\sindex[exhibitor]{Lennie, J. M. \& Co, Nearhouse, Tankerness} &
	\raggedright 7yo cow &
	\raggedright Balthayock Violet\sindex[beef]{Balthayock Violet} &
	\raggedright By Balthayock Oliver.
	\tabularnewline
\hline
	$1992$ &
	\raggedright J. M. Lennie \& Co, Nearhouse, Tankerness\sindex[exhibitor]{Lennie, J. M. \& Co, Nearhouse, Tankerness} &
	\raggedright 9yo cow in milk &
	\raggedright Little Gill Verity\sindex[beef]{Little Gill Verity} &
	\raggedright By Tattenhall Hiblot, out of Holehouse Rebecca.
	\tabularnewline
\hline
	$1993$ &
	\raggedright W. R. Baillie, Biggings, Toab\sindex[exhibitor]{Baillie, W. R., Biggings, Toab} &
	\raggedright 18mo heifer* &
	\raggedright Ballieston Helga\sindex[beef]{Ballieston Helga} &
	\raggedright By Ballieston Donald, out of Ballieston Catherine.
	\tabularnewline
\hline
	$1994$ &
	\raggedright W. R. Baillie, Biggings, Toab\sindex[exhibitor]{Baillie, W. R., Biggings, Toab} &
	\raggedright 18mo bull* &
	\raggedright Ballieston Howard\sindex[beef]{Ballieston Howard} &
	\raggedright By Cockrington Baron, out of Ballieston Frances.
	\tabularnewline
\hline
	$1995$ &
	\raggedright W. R. Baillie, Biggings, Toab\sindex[exhibitor]{Baillie, W. R., Biggings, Toab} &
	\raggedright 15mo heifer* &
	\raggedright Ballieston Jacqueline\sindex[beef]{Ballieston Jacqueline} &
	\multirow{2}{8cm}{By Fleets Rascal, out of Ballieston Donna.}
	\tabularnewline
\cline{1-4}
	$1996$ &
	\raggedright W. R. Baillie, Biggings, Toab\sindex[exhibitor]{Baillie, W. R., Biggings, Toab} &
	\raggedright 20mo heifer* &
	\raggedright Ballieston Jacqueline\sindex[beef]{Ballieston Jacqueline} &
	\tabularnewline
\hline
	$1997$ &
	\raggedright J. A. Scott, Howe, Harray\sindex[exhibitor]{Scott, J. A., Howe, Harray} &
	\raggedright 6yo bull &
	\raggedright Seawell Goutte\sindex[beef]{Seawell Goutte} &
	\raggedright By Seawell Caribbean; bought at Perth as a 13mo.
	\tabularnewline
\hline
	$1998$ &
	\raggedright G. T. \& S. Coghill, Muce, Birsay\sindex[exhibitor]{Coghill, G. T. \& S., Muce, Birsay} &
	\raggedright 5yo cow in milk &
	\raggedright New Moan Iris\sindex[beef]{New Moan Iris} &
	\raggedright By Balthayoak Dalesman, out of New Moan Emma; bought from Jim Isbister, New Moan, Harray, in October 1997.
	\tabularnewline
\hline
	$1999$ &
	\raggedright Garson Farms, Sandwick\sindex[exhibitor]{Garson Farms, Sandwick} &
	\raggedright 6yo cow* &
	\raggedright Garson Isla\sindex[beef]{Garson Isla} &
	\raggedright By Brampton Consort, out of Gredington Caniche.
	\tabularnewline
\hline
	$2000$ &
	\raggedright W. R. Baillie, Biggings, Toab\sindex[exhibitor]{Baillie, W. R., Biggings, Toab} &
	\raggedright 7yo dry cow* &
	\raggedright Ballieston Ingrid\sindex[beef]{Ballieston Ingrid} &
	\raggedright By Maerdy Victorious, out of Ballieston Ursula.
	\tabularnewline
\hline
	$2001$ &
	\multicolumn{4}{c|}{No Show due to Foot \& Mouth outbreak.}
	\tabularnewline
\hline
	$2002$ &
	\raggedright L. Spence, Spengar, Sandwick\sindex[exhibitor]{Spence, L., Spengar, Sandwick} &
	\raggedright 16mo bull* &
	\raggedright Spengar Sambhar\sindex[beef]{Spengar Sambhar} &
	\raggedright By Brampton Four Star, out of Mortimer's Lauren.
	\tabularnewline
\hline
	$2003$ &
	\raggedright Garson Farms, Sandwick\sindex[exhibitor]{Garson Farms, Sandwick} &
	\raggedright 1yo heifer &
	\raggedright Airdlin Treasure\sindex[beef]{Airdlin Treasure} &
	\raggedright By Airdlin Ploughman, out of Maerdy Limpet; bought at Perth February 2003.
	\tabularnewline
\hline
	$2004$ &
	\raggedright W. R. Baillie, Biggings, Toab\sindex[exhibitor]{Baillie, W. R., Biggings, Toab} &
	\raggedright 17mo heifer* &
	\raggedright Baillieston Topsy\sindex[beef]{Baillieston Topsy} &
	\raggedright By Moyness President, out of Ballieston Maiden.
	\tabularnewline
\hline
	$2005$ &
	\raggedright Messrs Davidson, Skaill, Sandwick\sindex[exhibitor]{Davidson, Messrs, Skaill, Sandwick} &
	\raggedright 16mo bull* &
	\raggedright Skaill Victor\sindex[beef]{Skaill Victor} &
	\raggedright By Offenbach, out of Rosette; supreme champion 2005.
	\tabularnewline
\hline
	$2006$ &
	\raggedright Messrs Davidson, Skaill, Sandwick\sindex[exhibitor]{Davidson, Messrs, Skaill, Sandwick} &
	\raggedright 16mo heifer* &
	\raggedright Skaill Abbie\sindex[beef]{Skaill Abbie} &
	\raggedright By Nectragerc, out of Perche; supreme champion, 2006; supreme champion at 2006 Dounby Show.
	\tabularnewline
\hline
	$2007$ &
	\multicolumn{4}{c|}{No Show due to Foot \& Mouth outbreak.}
	\tabularnewline
\hline
	$2008$ &
	\raggedright W. R. Baillie, Biggings, Toab\sindex[exhibitor]{Baillie, W. R., Biggins, Toab} &
	\raggedright 2yo heifer* &
	\raggedright Baillieston Anthea\sindex[beef]{Baillieston Anthea} &
	\raggedright By Mowbraypark Umpire, out of Baillieston Umbreta; supreme champion.
	\tabularnewline
\hline
	$2009$ &
	\raggedright W. R. Baillie, Biggings, Toab\sindex[exhibitor]{Baillie, W. R., Biggins, Toab} &
	\raggedright 5yo cow-in-milk* &
	\raggedright Baillieston Vera\sindex[beef]{Baillieston Anthea} &
	\raggedright By Moyness President, out of Baillieston Ola; supreme champion.
	\tabularnewline
\hline
	$2010$ &
	\raggedright J. M. Lennie \& Co, Nearhouse, Tankerness\sindex[exhibitor]{Lennie, J. M. \& Co, Nearhouse, Tankerness} &
	\raggedright 4yo bull* &
	\raggedright Nearhouse Bobby &
	\raggedright By Fremch bull Oscar, out of Brymire Ola.
	\tabularnewline
\hline
	$2011$ &
	\raggedright  &
	\raggedright  &
	\raggedright  &
	\raggedright 
	\tabularnewline
\hline
\end{longtable}

\normalsize

		\chapter{Beef Cattle}
\section[Supreme Champion]{'Hope Show Supreme Champion $1970 - 2009$}
\small
\begin{longtable}{|c|p{5.2cm}|p{3cm}|p{3cm}|p{8cm}|}
\hline
	\textbf{Year} &
	\textbf{Exhibitor} &
	\textbf{Champion} &
	\textbf{Name} &
	\textbf{Notes} 
	\tabularnewline
\hline
\endhead
	$1978$ &
	\raggedright W. R. Baillie, Biggings, Toab\sindex[exhibitor]{Baillie, W. R., Biggings, Toab} &
	\raggedright 4yo bull &
	\raggedright Wincote Jasper\sindex[beef]{Wincote Jasper} &
	\raggedright By Bent Caribineer; bred by Lord Nelson of Stafford; bought at Northampton for 950gns; supreme champion in 1978 \& 1982.
	\tabularnewline
\hline
	$1979$ &
	\multicolumn{4}{c|}{No entries forward.}
	\tabularnewline
\hline
	$1980$ &
	\raggedright W. R. Baillie, Biggings, Toab\sindex[exhibitor]{Baillie, W. R., Biggings, Toab} &
	\raggedright 6yo bull &
	\raggedright Wincote Jasper\sindex[beef]{Wincote Jasper} &
	\multirow{3}{8cm}{By Bent Caribineer; bred by Lord Nelson of Stafford; bought at Northampton for 950gns; supreme champion in 1978 \& 1982.}
	\tabularnewline
\cline{1-4}
	$1981$ &
	\raggedright W. R. Baillie, Biggings, Toab\sindex[exhibitor]{Baillie, W. R., Biggings, Toab} &
	\raggedright 7yo bull &
	\raggedright Wincote Jasper\sindex[beef]{Wincote Jasper} &
	\tabularnewline
\cline{1-4}
	$1982$ &
	\raggedright W. R. Baillie, Biggings, Toab\sindex[exhibitor]{Baillie, W. R., Biggings, Toab} &
	\raggedright 8yo bull &
	\raggedright Wincote Jasper\sindex[beef]{Wincote Jasper} &
	\tabularnewline
\hline
	$1983$ &
	\raggedright W. R. Baillie, Biggings, Toab\sindex[exhibitor]{Baillie, W. R., Biggings, Toab} &
	\raggedright - &
	\raggedright - &
	\raggedright -
	\tabularnewline
\hline
	$1984$ &
	\raggedright W. R. Baillie, Biggings, Toab\sindex[exhibitor]{Baillie, W. R., Biggings, Toab} &
	\raggedright 10mo bull* &
	\raggedright Ballieston Universe\sindex[beef]{Ballieston Universe} &
	\raggedright By Wincote Jasper, out of Sway Ninette.
	\tabularnewline
\hline
	$1985$ &
	\raggedright G. T. \& S. Coghill, Muce, Birsay\sindex[exhibitor]{Coghill, G. T. \& S., Muce, Birsay} &
	\raggedright 3yo cow in calf &
	\raggedright Mornview Tamar\sindex[beef]{Mornview Tamar} &
	\raggedright By Tattenhall Hublot, out of a Mornview cow.
	\tabularnewline
\hline
	$1986$ &
	\raggedright W. G. Kirkpatrick \& Sons, Newhall, Stromness\sindex[exhibitor]{Kirkpatrick, W. G. \& Sons, Newhall, Stromness} &
	\raggedright 3yo bull &
	\raggedright Ballieston Universe\sindex[beef]{Ballieston Universe} &
	\raggedright By Wincote Jasper, out of Sway Ninette.
	\tabularnewline
\hline
	$1987$ &
	\raggedright G. T. \& S. Coghill, Muce, Birsay\sindex[exhibitor]{Coghill, G. T. \& S., Muce, Birsay} &
	\raggedright 8mo bull &
	\raggedright Hekra Challenger\sindex[beef]{Hekra Challenger} &
	\raggedright By Fairthwaite Magic, out of Hekra Vera.
	\tabularnewline
\hline
	$1988$ &
	\raggedright W. R. Baillie, Biggings, Toab\sindex[exhibitor]{Baillie, W. R., Biggings, Toab} &
	\raggedright 4yo cow* &
	\raggedright Ballieston Vaila\sindex[beef]{Ballieston Vaila} &
	\raggedright By Lunsdale Harvest, out of Ballieston Snowflake.
	\tabularnewline
\hline
	$1989$ &
	\raggedright J. M. Lennie \& Co, Nearhouse, Tankerness\sindex[exhibitor]{Lennie, J. M. \& Co, Nearhouse, Tankerness} &
	\raggedright 5yo cow in calf &
	\raggedright Little Gill Verity\sindex[beef]{Little Gill Verity} &
	\raggedright By Tatton Hall Hublot, out of Hole Moor Rebecca.
	\tabularnewline
\hline
	$1990$ &
	\raggedright D. W. Sinclair \& Co, Unigarth, Sandwick\sindex[exhibitor]{} &
	\raggedright 2yo bull &
	\raggedright Royston Democrat\sindex[beef]{Royston Democrat} &
	\raggedright By Glenleary Attenhall.
	\tabularnewline
\hline
	$1991$ &
	\raggedright J. M. Lennie \& Co, Nearhouse, Tankerness\sindex[exhibitor]{Lennie, J. M. \& Co, Nearhouse, Tankerness} &
	\raggedright 7yo cow &
	\raggedright Balthayock Violet\sindex[beef]{Balthayock Violet} &
	\raggedright By Balthayock Oliver.
	\tabularnewline
\hline
	$1992$ &
	\raggedright J. M. Lennie \& Co, Nearhouse, Tankerness\sindex[exhibitor]{Lennie, J. M. \& Co, Nearhouse, Tankerness} &
	\raggedright 9yo cow in milk &
	\raggedright Little Gill Verity\sindex[beef]{Little Gill Verity} &
	\raggedright By Tattenhall Hiblot, out of Holehouse Rebecca.
	\tabularnewline
\hline
	$1993$ &
	\raggedright W. R. Baillie, Biggings, Toab\sindex[exhibitor]{Baillie, W. R., Biggings, Toab} &
	\raggedright 18mo heifer* &
	\raggedright Ballieston Helga\sindex[beef]{Ballieston Helga} &
	\raggedright By Ballieston Donald, out of Ballieston Catherine.
	\tabularnewline
\hline
	$1994$ &
	\raggedright W. R. Baillie, Biggings, Toab\sindex[exhibitor]{Baillie, W. R., Biggings, Toab} &
	\raggedright 18mo bull* &
	\raggedright Ballieston Howard\sindex[beef]{Ballieston Howard} &
	\raggedright By Cockrington Baron, out of Ballieston Frances.
	\tabularnewline
\hline
	$1995$ &
	\raggedright W. R. Baillie, Biggings, Toab\sindex[exhibitor]{Baillie, W. R., Biggings, Toab} &
	\raggedright 15mo heifer* &
	\raggedright Ballieston Jacqueline\sindex[beef]{Ballieston Jacqueline} &
	\multirow{2}{8cm}{By Fleets Rascal, out of Ballieston Donna.}
	\tabularnewline
\cline{1-4}
	$1996$ &
	\raggedright W. R. Baillie, Biggings, Toab\sindex[exhibitor]{Baillie, W. R., Biggings, Toab} &
	\raggedright 20mo heifer* &
	\raggedright Ballieston Jacqueline\sindex[beef]{Ballieston Jacqueline} &
	\tabularnewline
\hline
	$1997$ &
	\raggedright J. A. Scott, Howe, Harray\sindex[exhibitor]{Scott, J. A., Howe, Harray} &
	\raggedright 6yo bull &
	\raggedright Seawell Goutte\sindex[beef]{Seawell Goutte} &
	\raggedright By Seawell Caribbean; bought at Perth as a 13mo.
	\tabularnewline
\hline
	$1998$ &
	\raggedright G. T. \& S. Coghill, Muce, Birsay\sindex[exhibitor]{Coghill, G. T. \& S., Muce, Birsay} &
	\raggedright 5yo cow in milk &
	\raggedright New Moan Iris\sindex[beef]{New Moan Iris} &
	\raggedright By Balthayoak Dalesman, out of New Moan Emma; bought from Jim Isbister, New Moan, Harray, in October 1997.
	\tabularnewline
\hline
	$1999$ &
	\raggedright Garson Farms, Sandwick\sindex[exhibitor]{Garson Farms, Sandwick} &
	\raggedright 6yo cow* &
	\raggedright Garson Isla\sindex[beef]{Garson Isla} &
	\raggedright By Brampton Consort, out of Gredington Caniche.
	\tabularnewline
\hline
	$2000$ &
	\raggedright W. R. Baillie, Biggings, Toab\sindex[exhibitor]{Baillie, W. R., Biggings, Toab} &
	\raggedright 7yo dry cow* &
	\raggedright Ballieston Ingrid\sindex[beef]{Ballieston Ingrid} &
	\raggedright By Maerdy Victorious, out of Ballieston Ursula.
	\tabularnewline
\hline
	$2001$ &
	\multicolumn{4}{c|}{No Show due to Foot \& Mouth outbreak.}
	\tabularnewline
\hline
	$2002$ &
	\raggedright L. Spence, Spengar, Sandwick\sindex[exhibitor]{Spence, L., Spengar, Sandwick} &
	\raggedright 16mo bull* &
	\raggedright Spengar Sambhar\sindex[beef]{Spengar Sambhar} &
	\raggedright By Brampton Four Star, out of Mortimer's Lauren.
	\tabularnewline
\hline
	$2003$ &
	\raggedright Garson Farms, Sandwick\sindex[exhibitor]{Garson Farms, Sandwick} &
	\raggedright 1yo heifer &
	\raggedright Airdlin Treasure\sindex[beef]{Airdlin Treasure} &
	\raggedright By Airdlin Ploughman, out of Maerdy Limpet; bought at Perth February 2003.
	\tabularnewline
\hline
	$2004$ &
	\raggedright W. R. Baillie, Biggings, Toab\sindex[exhibitor]{Baillie, W. R., Biggings, Toab} &
	\raggedright 17mo heifer* &
	\raggedright Baillieston Topsy\sindex[beef]{Baillieston Topsy} &
	\raggedright By Moyness President, out of Ballieston Maiden.
	\tabularnewline
\hline
	$2005$ &
	\raggedright Messrs Davidson, Skaill, Sandwick\sindex[exhibitor]{Davidson, Messrs, Skaill, Sandwick} &
	\raggedright 16mo bull* &
	\raggedright Skaill Victor\sindex[beef]{Skaill Victor} &
	\raggedright By Offenbach, out of Rosette; supreme champion 2005.
	\tabularnewline
\hline
	$2006$ &
	\raggedright Messrs Davidson, Skaill, Sandwick\sindex[exhibitor]{Davidson, Messrs, Skaill, Sandwick} &
	\raggedright 16mo heifer* &
	\raggedright Skaill Abbie\sindex[beef]{Skaill Abbie} &
	\raggedright By Nectragerc, out of Perche; supreme champion, 2006; supreme champion at 2006 Dounby Show.
	\tabularnewline
\hline
	$2007$ &
	\multicolumn{4}{c|}{No Show due to Foot \& Mouth outbreak.}
	\tabularnewline
\hline
	$2008$ &
	\raggedright W. R. Baillie, Biggings, Toab\sindex[exhibitor]{Baillie, W. R., Biggins, Toab} &
	\raggedright 2yo heifer* &
	\raggedright Baillieston Anthea\sindex[beef]{Baillieston Anthea} &
	\raggedright By Mowbraypark Umpire, out of Baillieston Umbreta; supreme champion.
	\tabularnewline
\hline
	$2009$ &
	\raggedright W. R. Baillie, Biggings, Toab\sindex[exhibitor]{Baillie, W. R., Biggins, Toab} &
	\raggedright 5yo cow-in-milk* &
	\raggedright Baillieston Vera\sindex[beef]{Baillieston Anthea} &
	\raggedright By Moyness President, out of Baillieston Ola; supreme champion.
	\tabularnewline
\hline
	$2010$ &
	\raggedright J. M. Lennie \& Co, Nearhouse, Tankerness\sindex[exhibitor]{Lennie, J. M. \& Co, Nearhouse, Tankerness} &
	\raggedright 4yo bull* &
	\raggedright Nearhouse Bobby &
	\raggedright By Fremch bull Oscar, out of Brymire Ola.
	\tabularnewline
\hline
	$2011$ &
	\raggedright  &
	\raggedright  &
	\raggedright  &
	\raggedright 
	\tabularnewline
\hline
\end{longtable}

\normalsize

		\chapter{Beef Cattle}
\section[Supreme Champion]{'Hope Show Supreme Champion $1970 - 2009$}
\small
\begin{longtable}{|c|p{5.2cm}|p{3cm}|p{3cm}|p{8cm}|}
\hline
	\textbf{Year} &
	\textbf{Exhibitor} &
	\textbf{Champion} &
	\textbf{Name} &
	\textbf{Notes} 
	\tabularnewline
\hline
\endhead
	$1978$ &
	\raggedright W. R. Baillie, Biggings, Toab\sindex[exhibitor]{Baillie, W. R., Biggings, Toab} &
	\raggedright 4yo bull &
	\raggedright Wincote Jasper\sindex[beef]{Wincote Jasper} &
	\raggedright By Bent Caribineer; bred by Lord Nelson of Stafford; bought at Northampton for 950gns; supreme champion in 1978 \& 1982.
	\tabularnewline
\hline
	$1979$ &
	\multicolumn{4}{c|}{No entries forward.}
	\tabularnewline
\hline
	$1980$ &
	\raggedright W. R. Baillie, Biggings, Toab\sindex[exhibitor]{Baillie, W. R., Biggings, Toab} &
	\raggedright 6yo bull &
	\raggedright Wincote Jasper\sindex[beef]{Wincote Jasper} &
	\multirow{3}{8cm}{By Bent Caribineer; bred by Lord Nelson of Stafford; bought at Northampton for 950gns; supreme champion in 1978 \& 1982.}
	\tabularnewline
\cline{1-4}
	$1981$ &
	\raggedright W. R. Baillie, Biggings, Toab\sindex[exhibitor]{Baillie, W. R., Biggings, Toab} &
	\raggedright 7yo bull &
	\raggedright Wincote Jasper\sindex[beef]{Wincote Jasper} &
	\tabularnewline
\cline{1-4}
	$1982$ &
	\raggedright W. R. Baillie, Biggings, Toab\sindex[exhibitor]{Baillie, W. R., Biggings, Toab} &
	\raggedright 8yo bull &
	\raggedright Wincote Jasper\sindex[beef]{Wincote Jasper} &
	\tabularnewline
\hline
	$1983$ &
	\raggedright W. R. Baillie, Biggings, Toab\sindex[exhibitor]{Baillie, W. R., Biggings, Toab} &
	\raggedright - &
	\raggedright - &
	\raggedright -
	\tabularnewline
\hline
	$1984$ &
	\raggedright W. R. Baillie, Biggings, Toab\sindex[exhibitor]{Baillie, W. R., Biggings, Toab} &
	\raggedright 10mo bull* &
	\raggedright Ballieston Universe\sindex[beef]{Ballieston Universe} &
	\raggedright By Wincote Jasper, out of Sway Ninette.
	\tabularnewline
\hline
	$1985$ &
	\raggedright G. T. \& S. Coghill, Muce, Birsay\sindex[exhibitor]{Coghill, G. T. \& S., Muce, Birsay} &
	\raggedright 3yo cow in calf &
	\raggedright Mornview Tamar\sindex[beef]{Mornview Tamar} &
	\raggedright By Tattenhall Hublot, out of a Mornview cow.
	\tabularnewline
\hline
	$1986$ &
	\raggedright W. G. Kirkpatrick \& Sons, Newhall, Stromness\sindex[exhibitor]{Kirkpatrick, W. G. \& Sons, Newhall, Stromness} &
	\raggedright 3yo bull &
	\raggedright Ballieston Universe\sindex[beef]{Ballieston Universe} &
	\raggedright By Wincote Jasper, out of Sway Ninette.
	\tabularnewline
\hline
	$1987$ &
	\raggedright G. T. \& S. Coghill, Muce, Birsay\sindex[exhibitor]{Coghill, G. T. \& S., Muce, Birsay} &
	\raggedright 8mo bull &
	\raggedright Hekra Challenger\sindex[beef]{Hekra Challenger} &
	\raggedright By Fairthwaite Magic, out of Hekra Vera.
	\tabularnewline
\hline
	$1988$ &
	\raggedright W. R. Baillie, Biggings, Toab\sindex[exhibitor]{Baillie, W. R., Biggings, Toab} &
	\raggedright 4yo cow* &
	\raggedright Ballieston Vaila\sindex[beef]{Ballieston Vaila} &
	\raggedright By Lunsdale Harvest, out of Ballieston Snowflake.
	\tabularnewline
\hline
	$1989$ &
	\raggedright J. M. Lennie \& Co, Nearhouse, Tankerness\sindex[exhibitor]{Lennie, J. M. \& Co, Nearhouse, Tankerness} &
	\raggedright 5yo cow in calf &
	\raggedright Little Gill Verity\sindex[beef]{Little Gill Verity} &
	\raggedright By Tatton Hall Hublot, out of Hole Moor Rebecca.
	\tabularnewline
\hline
	$1990$ &
	\raggedright D. W. Sinclair \& Co, Unigarth, Sandwick\sindex[exhibitor]{} &
	\raggedright 2yo bull &
	\raggedright Royston Democrat\sindex[beef]{Royston Democrat} &
	\raggedright By Glenleary Attenhall.
	\tabularnewline
\hline
	$1991$ &
	\raggedright J. M. Lennie \& Co, Nearhouse, Tankerness\sindex[exhibitor]{Lennie, J. M. \& Co, Nearhouse, Tankerness} &
	\raggedright 7yo cow &
	\raggedright Balthayock Violet\sindex[beef]{Balthayock Violet} &
	\raggedright By Balthayock Oliver.
	\tabularnewline
\hline
	$1992$ &
	\raggedright J. M. Lennie \& Co, Nearhouse, Tankerness\sindex[exhibitor]{Lennie, J. M. \& Co, Nearhouse, Tankerness} &
	\raggedright 9yo cow in milk &
	\raggedright Little Gill Verity\sindex[beef]{Little Gill Verity} &
	\raggedright By Tattenhall Hiblot, out of Holehouse Rebecca.
	\tabularnewline
\hline
	$1993$ &
	\raggedright W. R. Baillie, Biggings, Toab\sindex[exhibitor]{Baillie, W. R., Biggings, Toab} &
	\raggedright 18mo heifer* &
	\raggedright Ballieston Helga\sindex[beef]{Ballieston Helga} &
	\raggedright By Ballieston Donald, out of Ballieston Catherine.
	\tabularnewline
\hline
	$1994$ &
	\raggedright W. R. Baillie, Biggings, Toab\sindex[exhibitor]{Baillie, W. R., Biggings, Toab} &
	\raggedright 18mo bull* &
	\raggedright Ballieston Howard\sindex[beef]{Ballieston Howard} &
	\raggedright By Cockrington Baron, out of Ballieston Frances.
	\tabularnewline
\hline
	$1995$ &
	\raggedright W. R. Baillie, Biggings, Toab\sindex[exhibitor]{Baillie, W. R., Biggings, Toab} &
	\raggedright 15mo heifer* &
	\raggedright Ballieston Jacqueline\sindex[beef]{Ballieston Jacqueline} &
	\multirow{2}{8cm}{By Fleets Rascal, out of Ballieston Donna.}
	\tabularnewline
\cline{1-4}
	$1996$ &
	\raggedright W. R. Baillie, Biggings, Toab\sindex[exhibitor]{Baillie, W. R., Biggings, Toab} &
	\raggedright 20mo heifer* &
	\raggedright Ballieston Jacqueline\sindex[beef]{Ballieston Jacqueline} &
	\tabularnewline
\hline
	$1997$ &
	\raggedright J. A. Scott, Howe, Harray\sindex[exhibitor]{Scott, J. A., Howe, Harray} &
	\raggedright 6yo bull &
	\raggedright Seawell Goutte\sindex[beef]{Seawell Goutte} &
	\raggedright By Seawell Caribbean; bought at Perth as a 13mo.
	\tabularnewline
\hline
	$1998$ &
	\raggedright G. T. \& S. Coghill, Muce, Birsay\sindex[exhibitor]{Coghill, G. T. \& S., Muce, Birsay} &
	\raggedright 5yo cow in milk &
	\raggedright New Moan Iris\sindex[beef]{New Moan Iris} &
	\raggedright By Balthayoak Dalesman, out of New Moan Emma; bought from Jim Isbister, New Moan, Harray, in October 1997.
	\tabularnewline
\hline
	$1999$ &
	\raggedright Garson Farms, Sandwick\sindex[exhibitor]{Garson Farms, Sandwick} &
	\raggedright 6yo cow* &
	\raggedright Garson Isla\sindex[beef]{Garson Isla} &
	\raggedright By Brampton Consort, out of Gredington Caniche.
	\tabularnewline
\hline
	$2000$ &
	\raggedright W. R. Baillie, Biggings, Toab\sindex[exhibitor]{Baillie, W. R., Biggings, Toab} &
	\raggedright 7yo dry cow* &
	\raggedright Ballieston Ingrid\sindex[beef]{Ballieston Ingrid} &
	\raggedright By Maerdy Victorious, out of Ballieston Ursula.
	\tabularnewline
\hline
	$2001$ &
	\multicolumn{4}{c|}{No Show due to Foot \& Mouth outbreak.}
	\tabularnewline
\hline
	$2002$ &
	\raggedright L. Spence, Spengar, Sandwick\sindex[exhibitor]{Spence, L., Spengar, Sandwick} &
	\raggedright 16mo bull* &
	\raggedright Spengar Sambhar\sindex[beef]{Spengar Sambhar} &
	\raggedright By Brampton Four Star, out of Mortimer's Lauren.
	\tabularnewline
\hline
	$2003$ &
	\raggedright Garson Farms, Sandwick\sindex[exhibitor]{Garson Farms, Sandwick} &
	\raggedright 1yo heifer &
	\raggedright Airdlin Treasure\sindex[beef]{Airdlin Treasure} &
	\raggedright By Airdlin Ploughman, out of Maerdy Limpet; bought at Perth February 2003.
	\tabularnewline
\hline
	$2004$ &
	\raggedright W. R. Baillie, Biggings, Toab\sindex[exhibitor]{Baillie, W. R., Biggings, Toab} &
	\raggedright 17mo heifer* &
	\raggedright Baillieston Topsy\sindex[beef]{Baillieston Topsy} &
	\raggedright By Moyness President, out of Ballieston Maiden.
	\tabularnewline
\hline
	$2005$ &
	\raggedright Messrs Davidson, Skaill, Sandwick\sindex[exhibitor]{Davidson, Messrs, Skaill, Sandwick} &
	\raggedright 16mo bull* &
	\raggedright Skaill Victor\sindex[beef]{Skaill Victor} &
	\raggedright By Offenbach, out of Rosette; supreme champion 2005.
	\tabularnewline
\hline
	$2006$ &
	\raggedright Messrs Davidson, Skaill, Sandwick\sindex[exhibitor]{Davidson, Messrs, Skaill, Sandwick} &
	\raggedright 16mo heifer* &
	\raggedright Skaill Abbie\sindex[beef]{Skaill Abbie} &
	\raggedright By Nectragerc, out of Perche; supreme champion, 2006; supreme champion at 2006 Dounby Show.
	\tabularnewline
\hline
	$2007$ &
	\multicolumn{4}{c|}{No Show due to Foot \& Mouth outbreak.}
	\tabularnewline
\hline
	$2008$ &
	\raggedright W. R. Baillie, Biggings, Toab\sindex[exhibitor]{Baillie, W. R., Biggins, Toab} &
	\raggedright 2yo heifer* &
	\raggedright Baillieston Anthea\sindex[beef]{Baillieston Anthea} &
	\raggedright By Mowbraypark Umpire, out of Baillieston Umbreta; supreme champion.
	\tabularnewline
\hline
	$2009$ &
	\raggedright W. R. Baillie, Biggings, Toab\sindex[exhibitor]{Baillie, W. R., Biggins, Toab} &
	\raggedright 5yo cow-in-milk* &
	\raggedright Baillieston Vera\sindex[beef]{Baillieston Anthea} &
	\raggedright By Moyness President, out of Baillieston Ola; supreme champion.
	\tabularnewline
\hline
	$2010$ &
	\raggedright J. M. Lennie \& Co, Nearhouse, Tankerness\sindex[exhibitor]{Lennie, J. M. \& Co, Nearhouse, Tankerness} &
	\raggedright 4yo bull* &
	\raggedright Nearhouse Bobby &
	\raggedright By Fremch bull Oscar, out of Brymire Ola.
	\tabularnewline
\hline
	$2011$ &
	\raggedright  &
	\raggedright  &
	\raggedright  &
	\raggedright 
	\tabularnewline
\hline
\end{longtable}

\normalsize

		\chapter{Beef Cattle}
\section[Supreme Champion]{'Hope Show Supreme Champion $1970 - 2009$}
\small
\begin{longtable}{|c|p{5.2cm}|p{3cm}|p{3cm}|p{8cm}|}
\hline
	\textbf{Year} &
	\textbf{Exhibitor} &
	\textbf{Champion} &
	\textbf{Name} &
	\textbf{Notes} 
	\tabularnewline
\hline
\endhead
	$1978$ &
	\raggedright W. R. Baillie, Biggings, Toab\sindex[exhibitor]{Baillie, W. R., Biggings, Toab} &
	\raggedright 4yo bull &
	\raggedright Wincote Jasper\sindex[beef]{Wincote Jasper} &
	\raggedright By Bent Caribineer; bred by Lord Nelson of Stafford; bought at Northampton for 950gns; supreme champion in 1978 \& 1982.
	\tabularnewline
\hline
	$1979$ &
	\multicolumn{4}{c|}{No entries forward.}
	\tabularnewline
\hline
	$1980$ &
	\raggedright W. R. Baillie, Biggings, Toab\sindex[exhibitor]{Baillie, W. R., Biggings, Toab} &
	\raggedright 6yo bull &
	\raggedright Wincote Jasper\sindex[beef]{Wincote Jasper} &
	\multirow{3}{8cm}{By Bent Caribineer; bred by Lord Nelson of Stafford; bought at Northampton for 950gns; supreme champion in 1978 \& 1982.}
	\tabularnewline
\cline{1-4}
	$1981$ &
	\raggedright W. R. Baillie, Biggings, Toab\sindex[exhibitor]{Baillie, W. R., Biggings, Toab} &
	\raggedright 7yo bull &
	\raggedright Wincote Jasper\sindex[beef]{Wincote Jasper} &
	\tabularnewline
\cline{1-4}
	$1982$ &
	\raggedright W. R. Baillie, Biggings, Toab\sindex[exhibitor]{Baillie, W. R., Biggings, Toab} &
	\raggedright 8yo bull &
	\raggedright Wincote Jasper\sindex[beef]{Wincote Jasper} &
	\tabularnewline
\hline
	$1983$ &
	\raggedright W. R. Baillie, Biggings, Toab\sindex[exhibitor]{Baillie, W. R., Biggings, Toab} &
	\raggedright - &
	\raggedright - &
	\raggedright -
	\tabularnewline
\hline
	$1984$ &
	\raggedright W. R. Baillie, Biggings, Toab\sindex[exhibitor]{Baillie, W. R., Biggings, Toab} &
	\raggedright 10mo bull* &
	\raggedright Ballieston Universe\sindex[beef]{Ballieston Universe} &
	\raggedright By Wincote Jasper, out of Sway Ninette.
	\tabularnewline
\hline
	$1985$ &
	\raggedright G. T. \& S. Coghill, Muce, Birsay\sindex[exhibitor]{Coghill, G. T. \& S., Muce, Birsay} &
	\raggedright 3yo cow in calf &
	\raggedright Mornview Tamar\sindex[beef]{Mornview Tamar} &
	\raggedright By Tattenhall Hublot, out of a Mornview cow.
	\tabularnewline
\hline
	$1986$ &
	\raggedright W. G. Kirkpatrick \& Sons, Newhall, Stromness\sindex[exhibitor]{Kirkpatrick, W. G. \& Sons, Newhall, Stromness} &
	\raggedright 3yo bull &
	\raggedright Ballieston Universe\sindex[beef]{Ballieston Universe} &
	\raggedright By Wincote Jasper, out of Sway Ninette.
	\tabularnewline
\hline
	$1987$ &
	\raggedright G. T. \& S. Coghill, Muce, Birsay\sindex[exhibitor]{Coghill, G. T. \& S., Muce, Birsay} &
	\raggedright 8mo bull &
	\raggedright Hekra Challenger\sindex[beef]{Hekra Challenger} &
	\raggedright By Fairthwaite Magic, out of Hekra Vera.
	\tabularnewline
\hline
	$1988$ &
	\raggedright W. R. Baillie, Biggings, Toab\sindex[exhibitor]{Baillie, W. R., Biggings, Toab} &
	\raggedright 4yo cow* &
	\raggedright Ballieston Vaila\sindex[beef]{Ballieston Vaila} &
	\raggedright By Lunsdale Harvest, out of Ballieston Snowflake.
	\tabularnewline
\hline
	$1989$ &
	\raggedright J. M. Lennie \& Co, Nearhouse, Tankerness\sindex[exhibitor]{Lennie, J. M. \& Co, Nearhouse, Tankerness} &
	\raggedright 5yo cow in calf &
	\raggedright Little Gill Verity\sindex[beef]{Little Gill Verity} &
	\raggedright By Tatton Hall Hublot, out of Hole Moor Rebecca.
	\tabularnewline
\hline
	$1990$ &
	\raggedright D. W. Sinclair \& Co, Unigarth, Sandwick\sindex[exhibitor]{} &
	\raggedright 2yo bull &
	\raggedright Royston Democrat\sindex[beef]{Royston Democrat} &
	\raggedright By Glenleary Attenhall.
	\tabularnewline
\hline
	$1991$ &
	\raggedright J. M. Lennie \& Co, Nearhouse, Tankerness\sindex[exhibitor]{Lennie, J. M. \& Co, Nearhouse, Tankerness} &
	\raggedright 7yo cow &
	\raggedright Balthayock Violet\sindex[beef]{Balthayock Violet} &
	\raggedright By Balthayock Oliver.
	\tabularnewline
\hline
	$1992$ &
	\raggedright J. M. Lennie \& Co, Nearhouse, Tankerness\sindex[exhibitor]{Lennie, J. M. \& Co, Nearhouse, Tankerness} &
	\raggedright 9yo cow in milk &
	\raggedright Little Gill Verity\sindex[beef]{Little Gill Verity} &
	\raggedright By Tattenhall Hiblot, out of Holehouse Rebecca.
	\tabularnewline
\hline
	$1993$ &
	\raggedright W. R. Baillie, Biggings, Toab\sindex[exhibitor]{Baillie, W. R., Biggings, Toab} &
	\raggedright 18mo heifer* &
	\raggedright Ballieston Helga\sindex[beef]{Ballieston Helga} &
	\raggedright By Ballieston Donald, out of Ballieston Catherine.
	\tabularnewline
\hline
	$1994$ &
	\raggedright W. R. Baillie, Biggings, Toab\sindex[exhibitor]{Baillie, W. R., Biggings, Toab} &
	\raggedright 18mo bull* &
	\raggedright Ballieston Howard\sindex[beef]{Ballieston Howard} &
	\raggedright By Cockrington Baron, out of Ballieston Frances.
	\tabularnewline
\hline
	$1995$ &
	\raggedright W. R. Baillie, Biggings, Toab\sindex[exhibitor]{Baillie, W. R., Biggings, Toab} &
	\raggedright 15mo heifer* &
	\raggedright Ballieston Jacqueline\sindex[beef]{Ballieston Jacqueline} &
	\multirow{2}{8cm}{By Fleets Rascal, out of Ballieston Donna.}
	\tabularnewline
\cline{1-4}
	$1996$ &
	\raggedright W. R. Baillie, Biggings, Toab\sindex[exhibitor]{Baillie, W. R., Biggings, Toab} &
	\raggedright 20mo heifer* &
	\raggedright Ballieston Jacqueline\sindex[beef]{Ballieston Jacqueline} &
	\tabularnewline
\hline
	$1997$ &
	\raggedright J. A. Scott, Howe, Harray\sindex[exhibitor]{Scott, J. A., Howe, Harray} &
	\raggedright 6yo bull &
	\raggedright Seawell Goutte\sindex[beef]{Seawell Goutte} &
	\raggedright By Seawell Caribbean; bought at Perth as a 13mo.
	\tabularnewline
\hline
	$1998$ &
	\raggedright G. T. \& S. Coghill, Muce, Birsay\sindex[exhibitor]{Coghill, G. T. \& S., Muce, Birsay} &
	\raggedright 5yo cow in milk &
	\raggedright New Moan Iris\sindex[beef]{New Moan Iris} &
	\raggedright By Balthayoak Dalesman, out of New Moan Emma; bought from Jim Isbister, New Moan, Harray, in October 1997.
	\tabularnewline
\hline
	$1999$ &
	\raggedright Garson Farms, Sandwick\sindex[exhibitor]{Garson Farms, Sandwick} &
	\raggedright 6yo cow* &
	\raggedright Garson Isla\sindex[beef]{Garson Isla} &
	\raggedright By Brampton Consort, out of Gredington Caniche.
	\tabularnewline
\hline
	$2000$ &
	\raggedright W. R. Baillie, Biggings, Toab\sindex[exhibitor]{Baillie, W. R., Biggings, Toab} &
	\raggedright 7yo dry cow* &
	\raggedright Ballieston Ingrid\sindex[beef]{Ballieston Ingrid} &
	\raggedright By Maerdy Victorious, out of Ballieston Ursula.
	\tabularnewline
\hline
	$2001$ &
	\multicolumn{4}{c|}{No Show due to Foot \& Mouth outbreak.}
	\tabularnewline
\hline
	$2002$ &
	\raggedright L. Spence, Spengar, Sandwick\sindex[exhibitor]{Spence, L., Spengar, Sandwick} &
	\raggedright 16mo bull* &
	\raggedright Spengar Sambhar\sindex[beef]{Spengar Sambhar} &
	\raggedright By Brampton Four Star, out of Mortimer's Lauren.
	\tabularnewline
\hline
	$2003$ &
	\raggedright Garson Farms, Sandwick\sindex[exhibitor]{Garson Farms, Sandwick} &
	\raggedright 1yo heifer &
	\raggedright Airdlin Treasure\sindex[beef]{Airdlin Treasure} &
	\raggedright By Airdlin Ploughman, out of Maerdy Limpet; bought at Perth February 2003.
	\tabularnewline
\hline
	$2004$ &
	\raggedright W. R. Baillie, Biggings, Toab\sindex[exhibitor]{Baillie, W. R., Biggings, Toab} &
	\raggedright 17mo heifer* &
	\raggedright Baillieston Topsy\sindex[beef]{Baillieston Topsy} &
	\raggedright By Moyness President, out of Ballieston Maiden.
	\tabularnewline
\hline
	$2005$ &
	\raggedright Messrs Davidson, Skaill, Sandwick\sindex[exhibitor]{Davidson, Messrs, Skaill, Sandwick} &
	\raggedright 16mo bull* &
	\raggedright Skaill Victor\sindex[beef]{Skaill Victor} &
	\raggedright By Offenbach, out of Rosette; supreme champion 2005.
	\tabularnewline
\hline
	$2006$ &
	\raggedright Messrs Davidson, Skaill, Sandwick\sindex[exhibitor]{Davidson, Messrs, Skaill, Sandwick} &
	\raggedright 16mo heifer* &
	\raggedright Skaill Abbie\sindex[beef]{Skaill Abbie} &
	\raggedright By Nectragerc, out of Perche; supreme champion, 2006; supreme champion at 2006 Dounby Show.
	\tabularnewline
\hline
	$2007$ &
	\multicolumn{4}{c|}{No Show due to Foot \& Mouth outbreak.}
	\tabularnewline
\hline
	$2008$ &
	\raggedright W. R. Baillie, Biggings, Toab\sindex[exhibitor]{Baillie, W. R., Biggins, Toab} &
	\raggedright 2yo heifer* &
	\raggedright Baillieston Anthea\sindex[beef]{Baillieston Anthea} &
	\raggedright By Mowbraypark Umpire, out of Baillieston Umbreta; supreme champion.
	\tabularnewline
\hline
	$2009$ &
	\raggedright W. R. Baillie, Biggings, Toab\sindex[exhibitor]{Baillie, W. R., Biggins, Toab} &
	\raggedright 5yo cow-in-milk* &
	\raggedright Baillieston Vera\sindex[beef]{Baillieston Anthea} &
	\raggedright By Moyness President, out of Baillieston Ola; supreme champion.
	\tabularnewline
\hline
	$2010$ &
	\raggedright J. M. Lennie \& Co, Nearhouse, Tankerness\sindex[exhibitor]{Lennie, J. M. \& Co, Nearhouse, Tankerness} &
	\raggedright 4yo bull* &
	\raggedright Nearhouse Bobby &
	\raggedright By Fremch bull Oscar, out of Brymire Ola.
	\tabularnewline
\hline
	$2011$ &
	\raggedright  &
	\raggedright  &
	\raggedright  &
	\raggedright 
	\tabularnewline
\hline
\end{longtable}

\normalsize

		\chapter{Beef Cattle}
\section[Supreme Champion]{'Hope Show Supreme Champion $1970 - 2009$}
\small
\begin{longtable}{|c|p{5.2cm}|p{3cm}|p{3cm}|p{8cm}|}
\hline
	\textbf{Year} &
	\textbf{Exhibitor} &
	\textbf{Champion} &
	\textbf{Name} &
	\textbf{Notes} 
	\tabularnewline
\hline
\endhead
	$1978$ &
	\raggedright W. R. Baillie, Biggings, Toab\sindex[exhibitor]{Baillie, W. R., Biggings, Toab} &
	\raggedright 4yo bull &
	\raggedright Wincote Jasper\sindex[beef]{Wincote Jasper} &
	\raggedright By Bent Caribineer; bred by Lord Nelson of Stafford; bought at Northampton for 950gns; supreme champion in 1978 \& 1982.
	\tabularnewline
\hline
	$1979$ &
	\multicolumn{4}{c|}{No entries forward.}
	\tabularnewline
\hline
	$1980$ &
	\raggedright W. R. Baillie, Biggings, Toab\sindex[exhibitor]{Baillie, W. R., Biggings, Toab} &
	\raggedright 6yo bull &
	\raggedright Wincote Jasper\sindex[beef]{Wincote Jasper} &
	\multirow{3}{8cm}{By Bent Caribineer; bred by Lord Nelson of Stafford; bought at Northampton for 950gns; supreme champion in 1978 \& 1982.}
	\tabularnewline
\cline{1-4}
	$1981$ &
	\raggedright W. R. Baillie, Biggings, Toab\sindex[exhibitor]{Baillie, W. R., Biggings, Toab} &
	\raggedright 7yo bull &
	\raggedright Wincote Jasper\sindex[beef]{Wincote Jasper} &
	\tabularnewline
\cline{1-4}
	$1982$ &
	\raggedright W. R. Baillie, Biggings, Toab\sindex[exhibitor]{Baillie, W. R., Biggings, Toab} &
	\raggedright 8yo bull &
	\raggedright Wincote Jasper\sindex[beef]{Wincote Jasper} &
	\tabularnewline
\hline
	$1983$ &
	\raggedright W. R. Baillie, Biggings, Toab\sindex[exhibitor]{Baillie, W. R., Biggings, Toab} &
	\raggedright - &
	\raggedright - &
	\raggedright -
	\tabularnewline
\hline
	$1984$ &
	\raggedright W. R. Baillie, Biggings, Toab\sindex[exhibitor]{Baillie, W. R., Biggings, Toab} &
	\raggedright 10mo bull* &
	\raggedright Ballieston Universe\sindex[beef]{Ballieston Universe} &
	\raggedright By Wincote Jasper, out of Sway Ninette.
	\tabularnewline
\hline
	$1985$ &
	\raggedright G. T. \& S. Coghill, Muce, Birsay\sindex[exhibitor]{Coghill, G. T. \& S., Muce, Birsay} &
	\raggedright 3yo cow in calf &
	\raggedright Mornview Tamar\sindex[beef]{Mornview Tamar} &
	\raggedright By Tattenhall Hublot, out of a Mornview cow.
	\tabularnewline
\hline
	$1986$ &
	\raggedright W. G. Kirkpatrick \& Sons, Newhall, Stromness\sindex[exhibitor]{Kirkpatrick, W. G. \& Sons, Newhall, Stromness} &
	\raggedright 3yo bull &
	\raggedright Ballieston Universe\sindex[beef]{Ballieston Universe} &
	\raggedright By Wincote Jasper, out of Sway Ninette.
	\tabularnewline
\hline
	$1987$ &
	\raggedright G. T. \& S. Coghill, Muce, Birsay\sindex[exhibitor]{Coghill, G. T. \& S., Muce, Birsay} &
	\raggedright 8mo bull &
	\raggedright Hekra Challenger\sindex[beef]{Hekra Challenger} &
	\raggedright By Fairthwaite Magic, out of Hekra Vera.
	\tabularnewline
\hline
	$1988$ &
	\raggedright W. R. Baillie, Biggings, Toab\sindex[exhibitor]{Baillie, W. R., Biggings, Toab} &
	\raggedright 4yo cow* &
	\raggedright Ballieston Vaila\sindex[beef]{Ballieston Vaila} &
	\raggedright By Lunsdale Harvest, out of Ballieston Snowflake.
	\tabularnewline
\hline
	$1989$ &
	\raggedright J. M. Lennie \& Co, Nearhouse, Tankerness\sindex[exhibitor]{Lennie, J. M. \& Co, Nearhouse, Tankerness} &
	\raggedright 5yo cow in calf &
	\raggedright Little Gill Verity\sindex[beef]{Little Gill Verity} &
	\raggedright By Tatton Hall Hublot, out of Hole Moor Rebecca.
	\tabularnewline
\hline
	$1990$ &
	\raggedright D. W. Sinclair \& Co, Unigarth, Sandwick\sindex[exhibitor]{} &
	\raggedright 2yo bull &
	\raggedright Royston Democrat\sindex[beef]{Royston Democrat} &
	\raggedright By Glenleary Attenhall.
	\tabularnewline
\hline
	$1991$ &
	\raggedright J. M. Lennie \& Co, Nearhouse, Tankerness\sindex[exhibitor]{Lennie, J. M. \& Co, Nearhouse, Tankerness} &
	\raggedright 7yo cow &
	\raggedright Balthayock Violet\sindex[beef]{Balthayock Violet} &
	\raggedright By Balthayock Oliver.
	\tabularnewline
\hline
	$1992$ &
	\raggedright J. M. Lennie \& Co, Nearhouse, Tankerness\sindex[exhibitor]{Lennie, J. M. \& Co, Nearhouse, Tankerness} &
	\raggedright 9yo cow in milk &
	\raggedright Little Gill Verity\sindex[beef]{Little Gill Verity} &
	\raggedright By Tattenhall Hiblot, out of Holehouse Rebecca.
	\tabularnewline
\hline
	$1993$ &
	\raggedright W. R. Baillie, Biggings, Toab\sindex[exhibitor]{Baillie, W. R., Biggings, Toab} &
	\raggedright 18mo heifer* &
	\raggedright Ballieston Helga\sindex[beef]{Ballieston Helga} &
	\raggedright By Ballieston Donald, out of Ballieston Catherine.
	\tabularnewline
\hline
	$1994$ &
	\raggedright W. R. Baillie, Biggings, Toab\sindex[exhibitor]{Baillie, W. R., Biggings, Toab} &
	\raggedright 18mo bull* &
	\raggedright Ballieston Howard\sindex[beef]{Ballieston Howard} &
	\raggedright By Cockrington Baron, out of Ballieston Frances.
	\tabularnewline
\hline
	$1995$ &
	\raggedright W. R. Baillie, Biggings, Toab\sindex[exhibitor]{Baillie, W. R., Biggings, Toab} &
	\raggedright 15mo heifer* &
	\raggedright Ballieston Jacqueline\sindex[beef]{Ballieston Jacqueline} &
	\multirow{2}{8cm}{By Fleets Rascal, out of Ballieston Donna.}
	\tabularnewline
\cline{1-4}
	$1996$ &
	\raggedright W. R. Baillie, Biggings, Toab\sindex[exhibitor]{Baillie, W. R., Biggings, Toab} &
	\raggedright 20mo heifer* &
	\raggedright Ballieston Jacqueline\sindex[beef]{Ballieston Jacqueline} &
	\tabularnewline
\hline
	$1997$ &
	\raggedright J. A. Scott, Howe, Harray\sindex[exhibitor]{Scott, J. A., Howe, Harray} &
	\raggedright 6yo bull &
	\raggedright Seawell Goutte\sindex[beef]{Seawell Goutte} &
	\raggedright By Seawell Caribbean; bought at Perth as a 13mo.
	\tabularnewline
\hline
	$1998$ &
	\raggedright G. T. \& S. Coghill, Muce, Birsay\sindex[exhibitor]{Coghill, G. T. \& S., Muce, Birsay} &
	\raggedright 5yo cow in milk &
	\raggedright New Moan Iris\sindex[beef]{New Moan Iris} &
	\raggedright By Balthayoak Dalesman, out of New Moan Emma; bought from Jim Isbister, New Moan, Harray, in October 1997.
	\tabularnewline
\hline
	$1999$ &
	\raggedright Garson Farms, Sandwick\sindex[exhibitor]{Garson Farms, Sandwick} &
	\raggedright 6yo cow* &
	\raggedright Garson Isla\sindex[beef]{Garson Isla} &
	\raggedright By Brampton Consort, out of Gredington Caniche.
	\tabularnewline
\hline
	$2000$ &
	\raggedright W. R. Baillie, Biggings, Toab\sindex[exhibitor]{Baillie, W. R., Biggings, Toab} &
	\raggedright 7yo dry cow* &
	\raggedright Ballieston Ingrid\sindex[beef]{Ballieston Ingrid} &
	\raggedright By Maerdy Victorious, out of Ballieston Ursula.
	\tabularnewline
\hline
	$2001$ &
	\multicolumn{4}{c|}{No Show due to Foot \& Mouth outbreak.}
	\tabularnewline
\hline
	$2002$ &
	\raggedright L. Spence, Spengar, Sandwick\sindex[exhibitor]{Spence, L., Spengar, Sandwick} &
	\raggedright 16mo bull* &
	\raggedright Spengar Sambhar\sindex[beef]{Spengar Sambhar} &
	\raggedright By Brampton Four Star, out of Mortimer's Lauren.
	\tabularnewline
\hline
	$2003$ &
	\raggedright Garson Farms, Sandwick\sindex[exhibitor]{Garson Farms, Sandwick} &
	\raggedright 1yo heifer &
	\raggedright Airdlin Treasure\sindex[beef]{Airdlin Treasure} &
	\raggedright By Airdlin Ploughman, out of Maerdy Limpet; bought at Perth February 2003.
	\tabularnewline
\hline
	$2004$ &
	\raggedright W. R. Baillie, Biggings, Toab\sindex[exhibitor]{Baillie, W. R., Biggings, Toab} &
	\raggedright 17mo heifer* &
	\raggedright Baillieston Topsy\sindex[beef]{Baillieston Topsy} &
	\raggedright By Moyness President, out of Ballieston Maiden.
	\tabularnewline
\hline
	$2005$ &
	\raggedright Messrs Davidson, Skaill, Sandwick\sindex[exhibitor]{Davidson, Messrs, Skaill, Sandwick} &
	\raggedright 16mo bull* &
	\raggedright Skaill Victor\sindex[beef]{Skaill Victor} &
	\raggedright By Offenbach, out of Rosette; supreme champion 2005.
	\tabularnewline
\hline
	$2006$ &
	\raggedright Messrs Davidson, Skaill, Sandwick\sindex[exhibitor]{Davidson, Messrs, Skaill, Sandwick} &
	\raggedright 16mo heifer* &
	\raggedright Skaill Abbie\sindex[beef]{Skaill Abbie} &
	\raggedright By Nectragerc, out of Perche; supreme champion, 2006; supreme champion at 2006 Dounby Show.
	\tabularnewline
\hline
	$2007$ &
	\multicolumn{4}{c|}{No Show due to Foot \& Mouth outbreak.}
	\tabularnewline
\hline
	$2008$ &
	\raggedright W. R. Baillie, Biggings, Toab\sindex[exhibitor]{Baillie, W. R., Biggins, Toab} &
	\raggedright 2yo heifer* &
	\raggedright Baillieston Anthea\sindex[beef]{Baillieston Anthea} &
	\raggedright By Mowbraypark Umpire, out of Baillieston Umbreta; supreme champion.
	\tabularnewline
\hline
	$2009$ &
	\raggedright W. R. Baillie, Biggings, Toab\sindex[exhibitor]{Baillie, W. R., Biggins, Toab} &
	\raggedright 5yo cow-in-milk* &
	\raggedright Baillieston Vera\sindex[beef]{Baillieston Anthea} &
	\raggedright By Moyness President, out of Baillieston Ola; supreme champion.
	\tabularnewline
\hline
	$2010$ &
	\raggedright J. M. Lennie \& Co, Nearhouse, Tankerness\sindex[exhibitor]{Lennie, J. M. \& Co, Nearhouse, Tankerness} &
	\raggedright 4yo bull* &
	\raggedright Nearhouse Bobby &
	\raggedright By Fremch bull Oscar, out of Brymire Ola.
	\tabularnewline
\hline
	$2011$ &
	\raggedright  &
	\raggedright  &
	\raggedright  &
	\raggedright 
	\tabularnewline
\hline
\end{longtable}

\normalsize

		\chapter{Beef Cattle}
\section[Supreme Champion]{'Hope Show Supreme Champion $1970 - 2009$}
\small
\begin{longtable}{|c|p{5.2cm}|p{3cm}|p{3cm}|p{8cm}|}
\hline
	\textbf{Year} &
	\textbf{Exhibitor} &
	\textbf{Champion} &
	\textbf{Name} &
	\textbf{Notes} 
	\tabularnewline
\hline
\endhead
	$1978$ &
	\raggedright W. R. Baillie, Biggings, Toab\sindex[exhibitor]{Baillie, W. R., Biggings, Toab} &
	\raggedright 4yo bull &
	\raggedright Wincote Jasper\sindex[beef]{Wincote Jasper} &
	\raggedright By Bent Caribineer; bred by Lord Nelson of Stafford; bought at Northampton for 950gns; supreme champion in 1978 \& 1982.
	\tabularnewline
\hline
	$1979$ &
	\multicolumn{4}{c|}{No entries forward.}
	\tabularnewline
\hline
	$1980$ &
	\raggedright W. R. Baillie, Biggings, Toab\sindex[exhibitor]{Baillie, W. R., Biggings, Toab} &
	\raggedright 6yo bull &
	\raggedright Wincote Jasper\sindex[beef]{Wincote Jasper} &
	\multirow{3}{8cm}{By Bent Caribineer; bred by Lord Nelson of Stafford; bought at Northampton for 950gns; supreme champion in 1978 \& 1982.}
	\tabularnewline
\cline{1-4}
	$1981$ &
	\raggedright W. R. Baillie, Biggings, Toab\sindex[exhibitor]{Baillie, W. R., Biggings, Toab} &
	\raggedright 7yo bull &
	\raggedright Wincote Jasper\sindex[beef]{Wincote Jasper} &
	\tabularnewline
\cline{1-4}
	$1982$ &
	\raggedright W. R. Baillie, Biggings, Toab\sindex[exhibitor]{Baillie, W. R., Biggings, Toab} &
	\raggedright 8yo bull &
	\raggedright Wincote Jasper\sindex[beef]{Wincote Jasper} &
	\tabularnewline
\hline
	$1983$ &
	\raggedright W. R. Baillie, Biggings, Toab\sindex[exhibitor]{Baillie, W. R., Biggings, Toab} &
	\raggedright - &
	\raggedright - &
	\raggedright -
	\tabularnewline
\hline
	$1984$ &
	\raggedright W. R. Baillie, Biggings, Toab\sindex[exhibitor]{Baillie, W. R., Biggings, Toab} &
	\raggedright 10mo bull* &
	\raggedright Ballieston Universe\sindex[beef]{Ballieston Universe} &
	\raggedright By Wincote Jasper, out of Sway Ninette.
	\tabularnewline
\hline
	$1985$ &
	\raggedright G. T. \& S. Coghill, Muce, Birsay\sindex[exhibitor]{Coghill, G. T. \& S., Muce, Birsay} &
	\raggedright 3yo cow in calf &
	\raggedright Mornview Tamar\sindex[beef]{Mornview Tamar} &
	\raggedright By Tattenhall Hublot, out of a Mornview cow.
	\tabularnewline
\hline
	$1986$ &
	\raggedright W. G. Kirkpatrick \& Sons, Newhall, Stromness\sindex[exhibitor]{Kirkpatrick, W. G. \& Sons, Newhall, Stromness} &
	\raggedright 3yo bull &
	\raggedright Ballieston Universe\sindex[beef]{Ballieston Universe} &
	\raggedright By Wincote Jasper, out of Sway Ninette.
	\tabularnewline
\hline
	$1987$ &
	\raggedright G. T. \& S. Coghill, Muce, Birsay\sindex[exhibitor]{Coghill, G. T. \& S., Muce, Birsay} &
	\raggedright 8mo bull &
	\raggedright Hekra Challenger\sindex[beef]{Hekra Challenger} &
	\raggedright By Fairthwaite Magic, out of Hekra Vera.
	\tabularnewline
\hline
	$1988$ &
	\raggedright W. R. Baillie, Biggings, Toab\sindex[exhibitor]{Baillie, W. R., Biggings, Toab} &
	\raggedright 4yo cow* &
	\raggedright Ballieston Vaila\sindex[beef]{Ballieston Vaila} &
	\raggedright By Lunsdale Harvest, out of Ballieston Snowflake.
	\tabularnewline
\hline
	$1989$ &
	\raggedright J. M. Lennie \& Co, Nearhouse, Tankerness\sindex[exhibitor]{Lennie, J. M. \& Co, Nearhouse, Tankerness} &
	\raggedright 5yo cow in calf &
	\raggedright Little Gill Verity\sindex[beef]{Little Gill Verity} &
	\raggedright By Tatton Hall Hublot, out of Hole Moor Rebecca.
	\tabularnewline
\hline
	$1990$ &
	\raggedright D. W. Sinclair \& Co, Unigarth, Sandwick\sindex[exhibitor]{} &
	\raggedright 2yo bull &
	\raggedright Royston Democrat\sindex[beef]{Royston Democrat} &
	\raggedright By Glenleary Attenhall.
	\tabularnewline
\hline
	$1991$ &
	\raggedright J. M. Lennie \& Co, Nearhouse, Tankerness\sindex[exhibitor]{Lennie, J. M. \& Co, Nearhouse, Tankerness} &
	\raggedright 7yo cow &
	\raggedright Balthayock Violet\sindex[beef]{Balthayock Violet} &
	\raggedright By Balthayock Oliver.
	\tabularnewline
\hline
	$1992$ &
	\raggedright J. M. Lennie \& Co, Nearhouse, Tankerness\sindex[exhibitor]{Lennie, J. M. \& Co, Nearhouse, Tankerness} &
	\raggedright 9yo cow in milk &
	\raggedright Little Gill Verity\sindex[beef]{Little Gill Verity} &
	\raggedright By Tattenhall Hiblot, out of Holehouse Rebecca.
	\tabularnewline
\hline
	$1993$ &
	\raggedright W. R. Baillie, Biggings, Toab\sindex[exhibitor]{Baillie, W. R., Biggings, Toab} &
	\raggedright 18mo heifer* &
	\raggedright Ballieston Helga\sindex[beef]{Ballieston Helga} &
	\raggedright By Ballieston Donald, out of Ballieston Catherine.
	\tabularnewline
\hline
	$1994$ &
	\raggedright W. R. Baillie, Biggings, Toab\sindex[exhibitor]{Baillie, W. R., Biggings, Toab} &
	\raggedright 18mo bull* &
	\raggedright Ballieston Howard\sindex[beef]{Ballieston Howard} &
	\raggedright By Cockrington Baron, out of Ballieston Frances.
	\tabularnewline
\hline
	$1995$ &
	\raggedright W. R. Baillie, Biggings, Toab\sindex[exhibitor]{Baillie, W. R., Biggings, Toab} &
	\raggedright 15mo heifer* &
	\raggedright Ballieston Jacqueline\sindex[beef]{Ballieston Jacqueline} &
	\multirow{2}{8cm}{By Fleets Rascal, out of Ballieston Donna.}
	\tabularnewline
\cline{1-4}
	$1996$ &
	\raggedright W. R. Baillie, Biggings, Toab\sindex[exhibitor]{Baillie, W. R., Biggings, Toab} &
	\raggedright 20mo heifer* &
	\raggedright Ballieston Jacqueline\sindex[beef]{Ballieston Jacqueline} &
	\tabularnewline
\hline
	$1997$ &
	\raggedright J. A. Scott, Howe, Harray\sindex[exhibitor]{Scott, J. A., Howe, Harray} &
	\raggedright 6yo bull &
	\raggedright Seawell Goutte\sindex[beef]{Seawell Goutte} &
	\raggedright By Seawell Caribbean; bought at Perth as a 13mo.
	\tabularnewline
\hline
	$1998$ &
	\raggedright G. T. \& S. Coghill, Muce, Birsay\sindex[exhibitor]{Coghill, G. T. \& S., Muce, Birsay} &
	\raggedright 5yo cow in milk &
	\raggedright New Moan Iris\sindex[beef]{New Moan Iris} &
	\raggedright By Balthayoak Dalesman, out of New Moan Emma; bought from Jim Isbister, New Moan, Harray, in October 1997.
	\tabularnewline
\hline
	$1999$ &
	\raggedright Garson Farms, Sandwick\sindex[exhibitor]{Garson Farms, Sandwick} &
	\raggedright 6yo cow* &
	\raggedright Garson Isla\sindex[beef]{Garson Isla} &
	\raggedright By Brampton Consort, out of Gredington Caniche.
	\tabularnewline
\hline
	$2000$ &
	\raggedright W. R. Baillie, Biggings, Toab\sindex[exhibitor]{Baillie, W. R., Biggings, Toab} &
	\raggedright 7yo dry cow* &
	\raggedright Ballieston Ingrid\sindex[beef]{Ballieston Ingrid} &
	\raggedright By Maerdy Victorious, out of Ballieston Ursula.
	\tabularnewline
\hline
	$2001$ &
	\multicolumn{4}{c|}{No Show due to Foot \& Mouth outbreak.}
	\tabularnewline
\hline
	$2002$ &
	\raggedright L. Spence, Spengar, Sandwick\sindex[exhibitor]{Spence, L., Spengar, Sandwick} &
	\raggedright 16mo bull* &
	\raggedright Spengar Sambhar\sindex[beef]{Spengar Sambhar} &
	\raggedright By Brampton Four Star, out of Mortimer's Lauren.
	\tabularnewline
\hline
	$2003$ &
	\raggedright Garson Farms, Sandwick\sindex[exhibitor]{Garson Farms, Sandwick} &
	\raggedright 1yo heifer &
	\raggedright Airdlin Treasure\sindex[beef]{Airdlin Treasure} &
	\raggedright By Airdlin Ploughman, out of Maerdy Limpet; bought at Perth February 2003.
	\tabularnewline
\hline
	$2004$ &
	\raggedright W. R. Baillie, Biggings, Toab\sindex[exhibitor]{Baillie, W. R., Biggings, Toab} &
	\raggedright 17mo heifer* &
	\raggedright Baillieston Topsy\sindex[beef]{Baillieston Topsy} &
	\raggedright By Moyness President, out of Ballieston Maiden.
	\tabularnewline
\hline
	$2005$ &
	\raggedright Messrs Davidson, Skaill, Sandwick\sindex[exhibitor]{Davidson, Messrs, Skaill, Sandwick} &
	\raggedright 16mo bull* &
	\raggedright Skaill Victor\sindex[beef]{Skaill Victor} &
	\raggedright By Offenbach, out of Rosette; supreme champion 2005.
	\tabularnewline
\hline
	$2006$ &
	\raggedright Messrs Davidson, Skaill, Sandwick\sindex[exhibitor]{Davidson, Messrs, Skaill, Sandwick} &
	\raggedright 16mo heifer* &
	\raggedright Skaill Abbie\sindex[beef]{Skaill Abbie} &
	\raggedright By Nectragerc, out of Perche; supreme champion, 2006; supreme champion at 2006 Dounby Show.
	\tabularnewline
\hline
	$2007$ &
	\multicolumn{4}{c|}{No Show due to Foot \& Mouth outbreak.}
	\tabularnewline
\hline
	$2008$ &
	\raggedright W. R. Baillie, Biggings, Toab\sindex[exhibitor]{Baillie, W. R., Biggins, Toab} &
	\raggedright 2yo heifer* &
	\raggedright Baillieston Anthea\sindex[beef]{Baillieston Anthea} &
	\raggedright By Mowbraypark Umpire, out of Baillieston Umbreta; supreme champion.
	\tabularnewline
\hline
	$2009$ &
	\raggedright W. R. Baillie, Biggings, Toab\sindex[exhibitor]{Baillie, W. R., Biggins, Toab} &
	\raggedright 5yo cow-in-milk* &
	\raggedright Baillieston Vera\sindex[beef]{Baillieston Anthea} &
	\raggedright By Moyness President, out of Baillieston Ola; supreme champion.
	\tabularnewline
\hline
	$2010$ &
	\raggedright J. M. Lennie \& Co, Nearhouse, Tankerness\sindex[exhibitor]{Lennie, J. M. \& Co, Nearhouse, Tankerness} &
	\raggedright 4yo bull* &
	\raggedright Nearhouse Bobby &
	\raggedright By Fremch bull Oscar, out of Brymire Ola.
	\tabularnewline
\hline
	$2011$ &
	\raggedright  &
	\raggedright  &
	\raggedright  &
	\raggedright 
	\tabularnewline
\hline
\end{longtable}

\normalsize

		\chapter{Beef Cattle}
\section[Supreme Champion]{'Hope Show Supreme Champion $1970 - 2009$}
\small
\begin{longtable}{|c|p{5.2cm}|p{3cm}|p{3cm}|p{8cm}|}
\hline
	\textbf{Year} &
	\textbf{Exhibitor} &
	\textbf{Champion} &
	\textbf{Name} &
	\textbf{Notes} 
	\tabularnewline
\hline
\endhead
	$1978$ &
	\raggedright W. R. Baillie, Biggings, Toab\sindex[exhibitor]{Baillie, W. R., Biggings, Toab} &
	\raggedright 4yo bull &
	\raggedright Wincote Jasper\sindex[beef]{Wincote Jasper} &
	\raggedright By Bent Caribineer; bred by Lord Nelson of Stafford; bought at Northampton for 950gns; supreme champion in 1978 \& 1982.
	\tabularnewline
\hline
	$1979$ &
	\multicolumn{4}{c|}{No entries forward.}
	\tabularnewline
\hline
	$1980$ &
	\raggedright W. R. Baillie, Biggings, Toab\sindex[exhibitor]{Baillie, W. R., Biggings, Toab} &
	\raggedright 6yo bull &
	\raggedright Wincote Jasper\sindex[beef]{Wincote Jasper} &
	\multirow{3}{8cm}{By Bent Caribineer; bred by Lord Nelson of Stafford; bought at Northampton for 950gns; supreme champion in 1978 \& 1982.}
	\tabularnewline
\cline{1-4}
	$1981$ &
	\raggedright W. R. Baillie, Biggings, Toab\sindex[exhibitor]{Baillie, W. R., Biggings, Toab} &
	\raggedright 7yo bull &
	\raggedright Wincote Jasper\sindex[beef]{Wincote Jasper} &
	\tabularnewline
\cline{1-4}
	$1982$ &
	\raggedright W. R. Baillie, Biggings, Toab\sindex[exhibitor]{Baillie, W. R., Biggings, Toab} &
	\raggedright 8yo bull &
	\raggedright Wincote Jasper\sindex[beef]{Wincote Jasper} &
	\tabularnewline
\hline
	$1983$ &
	\raggedright W. R. Baillie, Biggings, Toab\sindex[exhibitor]{Baillie, W. R., Biggings, Toab} &
	\raggedright - &
	\raggedright - &
	\raggedright -
	\tabularnewline
\hline
	$1984$ &
	\raggedright W. R. Baillie, Biggings, Toab\sindex[exhibitor]{Baillie, W. R., Biggings, Toab} &
	\raggedright 10mo bull* &
	\raggedright Ballieston Universe\sindex[beef]{Ballieston Universe} &
	\raggedright By Wincote Jasper, out of Sway Ninette.
	\tabularnewline
\hline
	$1985$ &
	\raggedright G. T. \& S. Coghill, Muce, Birsay\sindex[exhibitor]{Coghill, G. T. \& S., Muce, Birsay} &
	\raggedright 3yo cow in calf &
	\raggedright Mornview Tamar\sindex[beef]{Mornview Tamar} &
	\raggedright By Tattenhall Hublot, out of a Mornview cow.
	\tabularnewline
\hline
	$1986$ &
	\raggedright W. G. Kirkpatrick \& Sons, Newhall, Stromness\sindex[exhibitor]{Kirkpatrick, W. G. \& Sons, Newhall, Stromness} &
	\raggedright 3yo bull &
	\raggedright Ballieston Universe\sindex[beef]{Ballieston Universe} &
	\raggedright By Wincote Jasper, out of Sway Ninette.
	\tabularnewline
\hline
	$1987$ &
	\raggedright G. T. \& S. Coghill, Muce, Birsay\sindex[exhibitor]{Coghill, G. T. \& S., Muce, Birsay} &
	\raggedright 8mo bull &
	\raggedright Hekra Challenger\sindex[beef]{Hekra Challenger} &
	\raggedright By Fairthwaite Magic, out of Hekra Vera.
	\tabularnewline
\hline
	$1988$ &
	\raggedright W. R. Baillie, Biggings, Toab\sindex[exhibitor]{Baillie, W. R., Biggings, Toab} &
	\raggedright 4yo cow* &
	\raggedright Ballieston Vaila\sindex[beef]{Ballieston Vaila} &
	\raggedright By Lunsdale Harvest, out of Ballieston Snowflake.
	\tabularnewline
\hline
	$1989$ &
	\raggedright J. M. Lennie \& Co, Nearhouse, Tankerness\sindex[exhibitor]{Lennie, J. M. \& Co, Nearhouse, Tankerness} &
	\raggedright 5yo cow in calf &
	\raggedright Little Gill Verity\sindex[beef]{Little Gill Verity} &
	\raggedright By Tatton Hall Hublot, out of Hole Moor Rebecca.
	\tabularnewline
\hline
	$1990$ &
	\raggedright D. W. Sinclair \& Co, Unigarth, Sandwick\sindex[exhibitor]{} &
	\raggedright 2yo bull &
	\raggedright Royston Democrat\sindex[beef]{Royston Democrat} &
	\raggedright By Glenleary Attenhall.
	\tabularnewline
\hline
	$1991$ &
	\raggedright J. M. Lennie \& Co, Nearhouse, Tankerness\sindex[exhibitor]{Lennie, J. M. \& Co, Nearhouse, Tankerness} &
	\raggedright 7yo cow &
	\raggedright Balthayock Violet\sindex[beef]{Balthayock Violet} &
	\raggedright By Balthayock Oliver.
	\tabularnewline
\hline
	$1992$ &
	\raggedright J. M. Lennie \& Co, Nearhouse, Tankerness\sindex[exhibitor]{Lennie, J. M. \& Co, Nearhouse, Tankerness} &
	\raggedright 9yo cow in milk &
	\raggedright Little Gill Verity\sindex[beef]{Little Gill Verity} &
	\raggedright By Tattenhall Hiblot, out of Holehouse Rebecca.
	\tabularnewline
\hline
	$1993$ &
	\raggedright W. R. Baillie, Biggings, Toab\sindex[exhibitor]{Baillie, W. R., Biggings, Toab} &
	\raggedright 18mo heifer* &
	\raggedright Ballieston Helga\sindex[beef]{Ballieston Helga} &
	\raggedright By Ballieston Donald, out of Ballieston Catherine.
	\tabularnewline
\hline
	$1994$ &
	\raggedright W. R. Baillie, Biggings, Toab\sindex[exhibitor]{Baillie, W. R., Biggings, Toab} &
	\raggedright 18mo bull* &
	\raggedright Ballieston Howard\sindex[beef]{Ballieston Howard} &
	\raggedright By Cockrington Baron, out of Ballieston Frances.
	\tabularnewline
\hline
	$1995$ &
	\raggedright W. R. Baillie, Biggings, Toab\sindex[exhibitor]{Baillie, W. R., Biggings, Toab} &
	\raggedright 15mo heifer* &
	\raggedright Ballieston Jacqueline\sindex[beef]{Ballieston Jacqueline} &
	\multirow{2}{8cm}{By Fleets Rascal, out of Ballieston Donna.}
	\tabularnewline
\cline{1-4}
	$1996$ &
	\raggedright W. R. Baillie, Biggings, Toab\sindex[exhibitor]{Baillie, W. R., Biggings, Toab} &
	\raggedright 20mo heifer* &
	\raggedright Ballieston Jacqueline\sindex[beef]{Ballieston Jacqueline} &
	\tabularnewline
\hline
	$1997$ &
	\raggedright J. A. Scott, Howe, Harray\sindex[exhibitor]{Scott, J. A., Howe, Harray} &
	\raggedright 6yo bull &
	\raggedright Seawell Goutte\sindex[beef]{Seawell Goutte} &
	\raggedright By Seawell Caribbean; bought at Perth as a 13mo.
	\tabularnewline
\hline
	$1998$ &
	\raggedright G. T. \& S. Coghill, Muce, Birsay\sindex[exhibitor]{Coghill, G. T. \& S., Muce, Birsay} &
	\raggedright 5yo cow in milk &
	\raggedright New Moan Iris\sindex[beef]{New Moan Iris} &
	\raggedright By Balthayoak Dalesman, out of New Moan Emma; bought from Jim Isbister, New Moan, Harray, in October 1997.
	\tabularnewline
\hline
	$1999$ &
	\raggedright Garson Farms, Sandwick\sindex[exhibitor]{Garson Farms, Sandwick} &
	\raggedright 6yo cow* &
	\raggedright Garson Isla\sindex[beef]{Garson Isla} &
	\raggedright By Brampton Consort, out of Gredington Caniche.
	\tabularnewline
\hline
	$2000$ &
	\raggedright W. R. Baillie, Biggings, Toab\sindex[exhibitor]{Baillie, W. R., Biggings, Toab} &
	\raggedright 7yo dry cow* &
	\raggedright Ballieston Ingrid\sindex[beef]{Ballieston Ingrid} &
	\raggedright By Maerdy Victorious, out of Ballieston Ursula.
	\tabularnewline
\hline
	$2001$ &
	\multicolumn{4}{c|}{No Show due to Foot \& Mouth outbreak.}
	\tabularnewline
\hline
	$2002$ &
	\raggedright L. Spence, Spengar, Sandwick\sindex[exhibitor]{Spence, L., Spengar, Sandwick} &
	\raggedright 16mo bull* &
	\raggedright Spengar Sambhar\sindex[beef]{Spengar Sambhar} &
	\raggedright By Brampton Four Star, out of Mortimer's Lauren.
	\tabularnewline
\hline
	$2003$ &
	\raggedright Garson Farms, Sandwick\sindex[exhibitor]{Garson Farms, Sandwick} &
	\raggedright 1yo heifer &
	\raggedright Airdlin Treasure\sindex[beef]{Airdlin Treasure} &
	\raggedright By Airdlin Ploughman, out of Maerdy Limpet; bought at Perth February 2003.
	\tabularnewline
\hline
	$2004$ &
	\raggedright W. R. Baillie, Biggings, Toab\sindex[exhibitor]{Baillie, W. R., Biggings, Toab} &
	\raggedright 17mo heifer* &
	\raggedright Baillieston Topsy\sindex[beef]{Baillieston Topsy} &
	\raggedright By Moyness President, out of Ballieston Maiden.
	\tabularnewline
\hline
	$2005$ &
	\raggedright Messrs Davidson, Skaill, Sandwick\sindex[exhibitor]{Davidson, Messrs, Skaill, Sandwick} &
	\raggedright 16mo bull* &
	\raggedright Skaill Victor\sindex[beef]{Skaill Victor} &
	\raggedright By Offenbach, out of Rosette; supreme champion 2005.
	\tabularnewline
\hline
	$2006$ &
	\raggedright Messrs Davidson, Skaill, Sandwick\sindex[exhibitor]{Davidson, Messrs, Skaill, Sandwick} &
	\raggedright 16mo heifer* &
	\raggedright Skaill Abbie\sindex[beef]{Skaill Abbie} &
	\raggedright By Nectragerc, out of Perche; supreme champion, 2006; supreme champion at 2006 Dounby Show.
	\tabularnewline
\hline
	$2007$ &
	\multicolumn{4}{c|}{No Show due to Foot \& Mouth outbreak.}
	\tabularnewline
\hline
	$2008$ &
	\raggedright W. R. Baillie, Biggings, Toab\sindex[exhibitor]{Baillie, W. R., Biggins, Toab} &
	\raggedright 2yo heifer* &
	\raggedright Baillieston Anthea\sindex[beef]{Baillieston Anthea} &
	\raggedright By Mowbraypark Umpire, out of Baillieston Umbreta; supreme champion.
	\tabularnewline
\hline
	$2009$ &
	\raggedright W. R. Baillie, Biggings, Toab\sindex[exhibitor]{Baillie, W. R., Biggins, Toab} &
	\raggedright 5yo cow-in-milk* &
	\raggedright Baillieston Vera\sindex[beef]{Baillieston Anthea} &
	\raggedright By Moyness President, out of Baillieston Ola; supreme champion.
	\tabularnewline
\hline
	$2010$ &
	\raggedright J. M. Lennie \& Co, Nearhouse, Tankerness\sindex[exhibitor]{Lennie, J. M. \& Co, Nearhouse, Tankerness} &
	\raggedright 4yo bull* &
	\raggedright Nearhouse Bobby &
	\raggedright By Fremch bull Oscar, out of Brymire Ola.
	\tabularnewline
\hline
	$2011$ &
	\raggedright  &
	\raggedright  &
	\raggedright  &
	\raggedright 
	\tabularnewline
\hline
\end{longtable}

\normalsize

		\chapter{Beef Cattle}
\section[Supreme Champion]{'Hope Show Supreme Champion $1970 - 2009$}
\small
\begin{longtable}{|c|p{5.2cm}|p{3cm}|p{3cm}|p{8cm}|}
\hline
	\textbf{Year} &
	\textbf{Exhibitor} &
	\textbf{Champion} &
	\textbf{Name} &
	\textbf{Notes} 
	\tabularnewline
\hline
\endhead
	$1978$ &
	\raggedright W. R. Baillie, Biggings, Toab\sindex[exhibitor]{Baillie, W. R., Biggings, Toab} &
	\raggedright 4yo bull &
	\raggedright Wincote Jasper\sindex[beef]{Wincote Jasper} &
	\raggedright By Bent Caribineer; bred by Lord Nelson of Stafford; bought at Northampton for 950gns; supreme champion in 1978 \& 1982.
	\tabularnewline
\hline
	$1979$ &
	\multicolumn{4}{c|}{No entries forward.}
	\tabularnewline
\hline
	$1980$ &
	\raggedright W. R. Baillie, Biggings, Toab\sindex[exhibitor]{Baillie, W. R., Biggings, Toab} &
	\raggedright 6yo bull &
	\raggedright Wincote Jasper\sindex[beef]{Wincote Jasper} &
	\multirow{3}{8cm}{By Bent Caribineer; bred by Lord Nelson of Stafford; bought at Northampton for 950gns; supreme champion in 1978 \& 1982.}
	\tabularnewline
\cline{1-4}
	$1981$ &
	\raggedright W. R. Baillie, Biggings, Toab\sindex[exhibitor]{Baillie, W. R., Biggings, Toab} &
	\raggedright 7yo bull &
	\raggedright Wincote Jasper\sindex[beef]{Wincote Jasper} &
	\tabularnewline
\cline{1-4}
	$1982$ &
	\raggedright W. R. Baillie, Biggings, Toab\sindex[exhibitor]{Baillie, W. R., Biggings, Toab} &
	\raggedright 8yo bull &
	\raggedright Wincote Jasper\sindex[beef]{Wincote Jasper} &
	\tabularnewline
\hline
	$1983$ &
	\raggedright W. R. Baillie, Biggings, Toab\sindex[exhibitor]{Baillie, W. R., Biggings, Toab} &
	\raggedright - &
	\raggedright - &
	\raggedright -
	\tabularnewline
\hline
	$1984$ &
	\raggedright W. R. Baillie, Biggings, Toab\sindex[exhibitor]{Baillie, W. R., Biggings, Toab} &
	\raggedright 10mo bull* &
	\raggedright Ballieston Universe\sindex[beef]{Ballieston Universe} &
	\raggedright By Wincote Jasper, out of Sway Ninette.
	\tabularnewline
\hline
	$1985$ &
	\raggedright G. T. \& S. Coghill, Muce, Birsay\sindex[exhibitor]{Coghill, G. T. \& S., Muce, Birsay} &
	\raggedright 3yo cow in calf &
	\raggedright Mornview Tamar\sindex[beef]{Mornview Tamar} &
	\raggedright By Tattenhall Hublot, out of a Mornview cow.
	\tabularnewline
\hline
	$1986$ &
	\raggedright W. G. Kirkpatrick \& Sons, Newhall, Stromness\sindex[exhibitor]{Kirkpatrick, W. G. \& Sons, Newhall, Stromness} &
	\raggedright 3yo bull &
	\raggedright Ballieston Universe\sindex[beef]{Ballieston Universe} &
	\raggedright By Wincote Jasper, out of Sway Ninette.
	\tabularnewline
\hline
	$1987$ &
	\raggedright G. T. \& S. Coghill, Muce, Birsay\sindex[exhibitor]{Coghill, G. T. \& S., Muce, Birsay} &
	\raggedright 8mo bull &
	\raggedright Hekra Challenger\sindex[beef]{Hekra Challenger} &
	\raggedright By Fairthwaite Magic, out of Hekra Vera.
	\tabularnewline
\hline
	$1988$ &
	\raggedright W. R. Baillie, Biggings, Toab\sindex[exhibitor]{Baillie, W. R., Biggings, Toab} &
	\raggedright 4yo cow* &
	\raggedright Ballieston Vaila\sindex[beef]{Ballieston Vaila} &
	\raggedright By Lunsdale Harvest, out of Ballieston Snowflake.
	\tabularnewline
\hline
	$1989$ &
	\raggedright J. M. Lennie \& Co, Nearhouse, Tankerness\sindex[exhibitor]{Lennie, J. M. \& Co, Nearhouse, Tankerness} &
	\raggedright 5yo cow in calf &
	\raggedright Little Gill Verity\sindex[beef]{Little Gill Verity} &
	\raggedright By Tatton Hall Hublot, out of Hole Moor Rebecca.
	\tabularnewline
\hline
	$1990$ &
	\raggedright D. W. Sinclair \& Co, Unigarth, Sandwick\sindex[exhibitor]{} &
	\raggedright 2yo bull &
	\raggedright Royston Democrat\sindex[beef]{Royston Democrat} &
	\raggedright By Glenleary Attenhall.
	\tabularnewline
\hline
	$1991$ &
	\raggedright J. M. Lennie \& Co, Nearhouse, Tankerness\sindex[exhibitor]{Lennie, J. M. \& Co, Nearhouse, Tankerness} &
	\raggedright 7yo cow &
	\raggedright Balthayock Violet\sindex[beef]{Balthayock Violet} &
	\raggedright By Balthayock Oliver.
	\tabularnewline
\hline
	$1992$ &
	\raggedright J. M. Lennie \& Co, Nearhouse, Tankerness\sindex[exhibitor]{Lennie, J. M. \& Co, Nearhouse, Tankerness} &
	\raggedright 9yo cow in milk &
	\raggedright Little Gill Verity\sindex[beef]{Little Gill Verity} &
	\raggedright By Tattenhall Hiblot, out of Holehouse Rebecca.
	\tabularnewline
\hline
	$1993$ &
	\raggedright W. R. Baillie, Biggings, Toab\sindex[exhibitor]{Baillie, W. R., Biggings, Toab} &
	\raggedright 18mo heifer* &
	\raggedright Ballieston Helga\sindex[beef]{Ballieston Helga} &
	\raggedright By Ballieston Donald, out of Ballieston Catherine.
	\tabularnewline
\hline
	$1994$ &
	\raggedright W. R. Baillie, Biggings, Toab\sindex[exhibitor]{Baillie, W. R., Biggings, Toab} &
	\raggedright 18mo bull* &
	\raggedright Ballieston Howard\sindex[beef]{Ballieston Howard} &
	\raggedright By Cockrington Baron, out of Ballieston Frances.
	\tabularnewline
\hline
	$1995$ &
	\raggedright W. R. Baillie, Biggings, Toab\sindex[exhibitor]{Baillie, W. R., Biggings, Toab} &
	\raggedright 15mo heifer* &
	\raggedright Ballieston Jacqueline\sindex[beef]{Ballieston Jacqueline} &
	\multirow{2}{8cm}{By Fleets Rascal, out of Ballieston Donna.}
	\tabularnewline
\cline{1-4}
	$1996$ &
	\raggedright W. R. Baillie, Biggings, Toab\sindex[exhibitor]{Baillie, W. R., Biggings, Toab} &
	\raggedright 20mo heifer* &
	\raggedright Ballieston Jacqueline\sindex[beef]{Ballieston Jacqueline} &
	\tabularnewline
\hline
	$1997$ &
	\raggedright J. A. Scott, Howe, Harray\sindex[exhibitor]{Scott, J. A., Howe, Harray} &
	\raggedright 6yo bull &
	\raggedright Seawell Goutte\sindex[beef]{Seawell Goutte} &
	\raggedright By Seawell Caribbean; bought at Perth as a 13mo.
	\tabularnewline
\hline
	$1998$ &
	\raggedright G. T. \& S. Coghill, Muce, Birsay\sindex[exhibitor]{Coghill, G. T. \& S., Muce, Birsay} &
	\raggedright 5yo cow in milk &
	\raggedright New Moan Iris\sindex[beef]{New Moan Iris} &
	\raggedright By Balthayoak Dalesman, out of New Moan Emma; bought from Jim Isbister, New Moan, Harray, in October 1997.
	\tabularnewline
\hline
	$1999$ &
	\raggedright Garson Farms, Sandwick\sindex[exhibitor]{Garson Farms, Sandwick} &
	\raggedright 6yo cow* &
	\raggedright Garson Isla\sindex[beef]{Garson Isla} &
	\raggedright By Brampton Consort, out of Gredington Caniche.
	\tabularnewline
\hline
	$2000$ &
	\raggedright W. R. Baillie, Biggings, Toab\sindex[exhibitor]{Baillie, W. R., Biggings, Toab} &
	\raggedright 7yo dry cow* &
	\raggedright Ballieston Ingrid\sindex[beef]{Ballieston Ingrid} &
	\raggedright By Maerdy Victorious, out of Ballieston Ursula.
	\tabularnewline
\hline
	$2001$ &
	\multicolumn{4}{c|}{No Show due to Foot \& Mouth outbreak.}
	\tabularnewline
\hline
	$2002$ &
	\raggedright L. Spence, Spengar, Sandwick\sindex[exhibitor]{Spence, L., Spengar, Sandwick} &
	\raggedright 16mo bull* &
	\raggedright Spengar Sambhar\sindex[beef]{Spengar Sambhar} &
	\raggedright By Brampton Four Star, out of Mortimer's Lauren.
	\tabularnewline
\hline
	$2003$ &
	\raggedright Garson Farms, Sandwick\sindex[exhibitor]{Garson Farms, Sandwick} &
	\raggedright 1yo heifer &
	\raggedright Airdlin Treasure\sindex[beef]{Airdlin Treasure} &
	\raggedright By Airdlin Ploughman, out of Maerdy Limpet; bought at Perth February 2003.
	\tabularnewline
\hline
	$2004$ &
	\raggedright W. R. Baillie, Biggings, Toab\sindex[exhibitor]{Baillie, W. R., Biggings, Toab} &
	\raggedright 17mo heifer* &
	\raggedright Baillieston Topsy\sindex[beef]{Baillieston Topsy} &
	\raggedright By Moyness President, out of Ballieston Maiden.
	\tabularnewline
\hline
	$2005$ &
	\raggedright Messrs Davidson, Skaill, Sandwick\sindex[exhibitor]{Davidson, Messrs, Skaill, Sandwick} &
	\raggedright 16mo bull* &
	\raggedright Skaill Victor\sindex[beef]{Skaill Victor} &
	\raggedright By Offenbach, out of Rosette; supreme champion 2005.
	\tabularnewline
\hline
	$2006$ &
	\raggedright Messrs Davidson, Skaill, Sandwick\sindex[exhibitor]{Davidson, Messrs, Skaill, Sandwick} &
	\raggedright 16mo heifer* &
	\raggedright Skaill Abbie\sindex[beef]{Skaill Abbie} &
	\raggedright By Nectragerc, out of Perche; supreme champion, 2006; supreme champion at 2006 Dounby Show.
	\tabularnewline
\hline
	$2007$ &
	\multicolumn{4}{c|}{No Show due to Foot \& Mouth outbreak.}
	\tabularnewline
\hline
	$2008$ &
	\raggedright W. R. Baillie, Biggings, Toab\sindex[exhibitor]{Baillie, W. R., Biggins, Toab} &
	\raggedright 2yo heifer* &
	\raggedright Baillieston Anthea\sindex[beef]{Baillieston Anthea} &
	\raggedright By Mowbraypark Umpire, out of Baillieston Umbreta; supreme champion.
	\tabularnewline
\hline
	$2009$ &
	\raggedright W. R. Baillie, Biggings, Toab\sindex[exhibitor]{Baillie, W. R., Biggins, Toab} &
	\raggedright 5yo cow-in-milk* &
	\raggedright Baillieston Vera\sindex[beef]{Baillieston Anthea} &
	\raggedright By Moyness President, out of Baillieston Ola; supreme champion.
	\tabularnewline
\hline
	$2010$ &
	\raggedright J. M. Lennie \& Co, Nearhouse, Tankerness\sindex[exhibitor]{Lennie, J. M. \& Co, Nearhouse, Tankerness} &
	\raggedright 4yo bull* &
	\raggedright Nearhouse Bobby &
	\raggedright By Fremch bull Oscar, out of Brymire Ola.
	\tabularnewline
\hline
	$2011$ &
	\raggedright  &
	\raggedright  &
	\raggedright  &
	\raggedright 
	\tabularnewline
\hline
\end{longtable}

\normalsize

		\chapter{Beef Cattle}
\section[Supreme Champion]{'Hope Show Supreme Champion $1970 - 2009$}
\small
\begin{longtable}{|c|p{5.2cm}|p{3cm}|p{3cm}|p{8cm}|}
\hline
	\textbf{Year} &
	\textbf{Exhibitor} &
	\textbf{Champion} &
	\textbf{Name} &
	\textbf{Notes} 
	\tabularnewline
\hline
\endhead
	$1978$ &
	\raggedright W. R. Baillie, Biggings, Toab\sindex[exhibitor]{Baillie, W. R., Biggings, Toab} &
	\raggedright 4yo bull &
	\raggedright Wincote Jasper\sindex[beef]{Wincote Jasper} &
	\raggedright By Bent Caribineer; bred by Lord Nelson of Stafford; bought at Northampton for 950gns; supreme champion in 1978 \& 1982.
	\tabularnewline
\hline
	$1979$ &
	\multicolumn{4}{c|}{No entries forward.}
	\tabularnewline
\hline
	$1980$ &
	\raggedright W. R. Baillie, Biggings, Toab\sindex[exhibitor]{Baillie, W. R., Biggings, Toab} &
	\raggedright 6yo bull &
	\raggedright Wincote Jasper\sindex[beef]{Wincote Jasper} &
	\multirow{3}{8cm}{By Bent Caribineer; bred by Lord Nelson of Stafford; bought at Northampton for 950gns; supreme champion in 1978 \& 1982.}
	\tabularnewline
\cline{1-4}
	$1981$ &
	\raggedright W. R. Baillie, Biggings, Toab\sindex[exhibitor]{Baillie, W. R., Biggings, Toab} &
	\raggedright 7yo bull &
	\raggedright Wincote Jasper\sindex[beef]{Wincote Jasper} &
	\tabularnewline
\cline{1-4}
	$1982$ &
	\raggedright W. R. Baillie, Biggings, Toab\sindex[exhibitor]{Baillie, W. R., Biggings, Toab} &
	\raggedright 8yo bull &
	\raggedright Wincote Jasper\sindex[beef]{Wincote Jasper} &
	\tabularnewline
\hline
	$1983$ &
	\raggedright W. R. Baillie, Biggings, Toab\sindex[exhibitor]{Baillie, W. R., Biggings, Toab} &
	\raggedright - &
	\raggedright - &
	\raggedright -
	\tabularnewline
\hline
	$1984$ &
	\raggedright W. R. Baillie, Biggings, Toab\sindex[exhibitor]{Baillie, W. R., Biggings, Toab} &
	\raggedright 10mo bull* &
	\raggedright Ballieston Universe\sindex[beef]{Ballieston Universe} &
	\raggedright By Wincote Jasper, out of Sway Ninette.
	\tabularnewline
\hline
	$1985$ &
	\raggedright G. T. \& S. Coghill, Muce, Birsay\sindex[exhibitor]{Coghill, G. T. \& S., Muce, Birsay} &
	\raggedright 3yo cow in calf &
	\raggedright Mornview Tamar\sindex[beef]{Mornview Tamar} &
	\raggedright By Tattenhall Hublot, out of a Mornview cow.
	\tabularnewline
\hline
	$1986$ &
	\raggedright W. G. Kirkpatrick \& Sons, Newhall, Stromness\sindex[exhibitor]{Kirkpatrick, W. G. \& Sons, Newhall, Stromness} &
	\raggedright 3yo bull &
	\raggedright Ballieston Universe\sindex[beef]{Ballieston Universe} &
	\raggedright By Wincote Jasper, out of Sway Ninette.
	\tabularnewline
\hline
	$1987$ &
	\raggedright G. T. \& S. Coghill, Muce, Birsay\sindex[exhibitor]{Coghill, G. T. \& S., Muce, Birsay} &
	\raggedright 8mo bull &
	\raggedright Hekra Challenger\sindex[beef]{Hekra Challenger} &
	\raggedright By Fairthwaite Magic, out of Hekra Vera.
	\tabularnewline
\hline
	$1988$ &
	\raggedright W. R. Baillie, Biggings, Toab\sindex[exhibitor]{Baillie, W. R., Biggings, Toab} &
	\raggedright 4yo cow* &
	\raggedright Ballieston Vaila\sindex[beef]{Ballieston Vaila} &
	\raggedright By Lunsdale Harvest, out of Ballieston Snowflake.
	\tabularnewline
\hline
	$1989$ &
	\raggedright J. M. Lennie \& Co, Nearhouse, Tankerness\sindex[exhibitor]{Lennie, J. M. \& Co, Nearhouse, Tankerness} &
	\raggedright 5yo cow in calf &
	\raggedright Little Gill Verity\sindex[beef]{Little Gill Verity} &
	\raggedright By Tatton Hall Hublot, out of Hole Moor Rebecca.
	\tabularnewline
\hline
	$1990$ &
	\raggedright D. W. Sinclair \& Co, Unigarth, Sandwick\sindex[exhibitor]{} &
	\raggedright 2yo bull &
	\raggedright Royston Democrat\sindex[beef]{Royston Democrat} &
	\raggedright By Glenleary Attenhall.
	\tabularnewline
\hline
	$1991$ &
	\raggedright J. M. Lennie \& Co, Nearhouse, Tankerness\sindex[exhibitor]{Lennie, J. M. \& Co, Nearhouse, Tankerness} &
	\raggedright 7yo cow &
	\raggedright Balthayock Violet\sindex[beef]{Balthayock Violet} &
	\raggedright By Balthayock Oliver.
	\tabularnewline
\hline
	$1992$ &
	\raggedright J. M. Lennie \& Co, Nearhouse, Tankerness\sindex[exhibitor]{Lennie, J. M. \& Co, Nearhouse, Tankerness} &
	\raggedright 9yo cow in milk &
	\raggedright Little Gill Verity\sindex[beef]{Little Gill Verity} &
	\raggedright By Tattenhall Hiblot, out of Holehouse Rebecca.
	\tabularnewline
\hline
	$1993$ &
	\raggedright W. R. Baillie, Biggings, Toab\sindex[exhibitor]{Baillie, W. R., Biggings, Toab} &
	\raggedright 18mo heifer* &
	\raggedright Ballieston Helga\sindex[beef]{Ballieston Helga} &
	\raggedright By Ballieston Donald, out of Ballieston Catherine.
	\tabularnewline
\hline
	$1994$ &
	\raggedright W. R. Baillie, Biggings, Toab\sindex[exhibitor]{Baillie, W. R., Biggings, Toab} &
	\raggedright 18mo bull* &
	\raggedright Ballieston Howard\sindex[beef]{Ballieston Howard} &
	\raggedright By Cockrington Baron, out of Ballieston Frances.
	\tabularnewline
\hline
	$1995$ &
	\raggedright W. R. Baillie, Biggings, Toab\sindex[exhibitor]{Baillie, W. R., Biggings, Toab} &
	\raggedright 15mo heifer* &
	\raggedright Ballieston Jacqueline\sindex[beef]{Ballieston Jacqueline} &
	\multirow{2}{8cm}{By Fleets Rascal, out of Ballieston Donna.}
	\tabularnewline
\cline{1-4}
	$1996$ &
	\raggedright W. R. Baillie, Biggings, Toab\sindex[exhibitor]{Baillie, W. R., Biggings, Toab} &
	\raggedright 20mo heifer* &
	\raggedright Ballieston Jacqueline\sindex[beef]{Ballieston Jacqueline} &
	\tabularnewline
\hline
	$1997$ &
	\raggedright J. A. Scott, Howe, Harray\sindex[exhibitor]{Scott, J. A., Howe, Harray} &
	\raggedright 6yo bull &
	\raggedright Seawell Goutte\sindex[beef]{Seawell Goutte} &
	\raggedright By Seawell Caribbean; bought at Perth as a 13mo.
	\tabularnewline
\hline
	$1998$ &
	\raggedright G. T. \& S. Coghill, Muce, Birsay\sindex[exhibitor]{Coghill, G. T. \& S., Muce, Birsay} &
	\raggedright 5yo cow in milk &
	\raggedright New Moan Iris\sindex[beef]{New Moan Iris} &
	\raggedright By Balthayoak Dalesman, out of New Moan Emma; bought from Jim Isbister, New Moan, Harray, in October 1997.
	\tabularnewline
\hline
	$1999$ &
	\raggedright Garson Farms, Sandwick\sindex[exhibitor]{Garson Farms, Sandwick} &
	\raggedright 6yo cow* &
	\raggedright Garson Isla\sindex[beef]{Garson Isla} &
	\raggedright By Brampton Consort, out of Gredington Caniche.
	\tabularnewline
\hline
	$2000$ &
	\raggedright W. R. Baillie, Biggings, Toab\sindex[exhibitor]{Baillie, W. R., Biggings, Toab} &
	\raggedright 7yo dry cow* &
	\raggedright Ballieston Ingrid\sindex[beef]{Ballieston Ingrid} &
	\raggedright By Maerdy Victorious, out of Ballieston Ursula.
	\tabularnewline
\hline
	$2001$ &
	\multicolumn{4}{c|}{No Show due to Foot \& Mouth outbreak.}
	\tabularnewline
\hline
	$2002$ &
	\raggedright L. Spence, Spengar, Sandwick\sindex[exhibitor]{Spence, L., Spengar, Sandwick} &
	\raggedright 16mo bull* &
	\raggedright Spengar Sambhar\sindex[beef]{Spengar Sambhar} &
	\raggedright By Brampton Four Star, out of Mortimer's Lauren.
	\tabularnewline
\hline
	$2003$ &
	\raggedright Garson Farms, Sandwick\sindex[exhibitor]{Garson Farms, Sandwick} &
	\raggedright 1yo heifer &
	\raggedright Airdlin Treasure\sindex[beef]{Airdlin Treasure} &
	\raggedright By Airdlin Ploughman, out of Maerdy Limpet; bought at Perth February 2003.
	\tabularnewline
\hline
	$2004$ &
	\raggedright W. R. Baillie, Biggings, Toab\sindex[exhibitor]{Baillie, W. R., Biggings, Toab} &
	\raggedright 17mo heifer* &
	\raggedright Baillieston Topsy\sindex[beef]{Baillieston Topsy} &
	\raggedright By Moyness President, out of Ballieston Maiden.
	\tabularnewline
\hline
	$2005$ &
	\raggedright Messrs Davidson, Skaill, Sandwick\sindex[exhibitor]{Davidson, Messrs, Skaill, Sandwick} &
	\raggedright 16mo bull* &
	\raggedright Skaill Victor\sindex[beef]{Skaill Victor} &
	\raggedright By Offenbach, out of Rosette; supreme champion 2005.
	\tabularnewline
\hline
	$2006$ &
	\raggedright Messrs Davidson, Skaill, Sandwick\sindex[exhibitor]{Davidson, Messrs, Skaill, Sandwick} &
	\raggedright 16mo heifer* &
	\raggedright Skaill Abbie\sindex[beef]{Skaill Abbie} &
	\raggedright By Nectragerc, out of Perche; supreme champion, 2006; supreme champion at 2006 Dounby Show.
	\tabularnewline
\hline
	$2007$ &
	\multicolumn{4}{c|}{No Show due to Foot \& Mouth outbreak.}
	\tabularnewline
\hline
	$2008$ &
	\raggedright W. R. Baillie, Biggings, Toab\sindex[exhibitor]{Baillie, W. R., Biggins, Toab} &
	\raggedright 2yo heifer* &
	\raggedright Baillieston Anthea\sindex[beef]{Baillieston Anthea} &
	\raggedright By Mowbraypark Umpire, out of Baillieston Umbreta; supreme champion.
	\tabularnewline
\hline
	$2009$ &
	\raggedright W. R. Baillie, Biggings, Toab\sindex[exhibitor]{Baillie, W. R., Biggins, Toab} &
	\raggedright 5yo cow-in-milk* &
	\raggedright Baillieston Vera\sindex[beef]{Baillieston Anthea} &
	\raggedright By Moyness President, out of Baillieston Ola; supreme champion.
	\tabularnewline
\hline
	$2010$ &
	\raggedright J. M. Lennie \& Co, Nearhouse, Tankerness\sindex[exhibitor]{Lennie, J. M. \& Co, Nearhouse, Tankerness} &
	\raggedright 4yo bull* &
	\raggedright Nearhouse Bobby &
	\raggedright By Fremch bull Oscar, out of Brymire Ola.
	\tabularnewline
\hline
	$2011$ &
	\raggedright  &
	\raggedright  &
	\raggedright  &
	\raggedright 
	\tabularnewline
\hline
\end{longtable}

\normalsize

	\end{landscape}
        \chapter{Beef Cattle}
\section[Supreme Champion]{'Hope Show Supreme Champion $1970 - 2009$}
\small
\begin{longtable}{|c|p{5.2cm}|p{3cm}|p{3cm}|p{8cm}|}
\hline
	\textbf{Year} &
	\textbf{Exhibitor} &
	\textbf{Champion} &
	\textbf{Name} &
	\textbf{Notes} 
	\tabularnewline
\hline
\endhead
	$1978$ &
	\raggedright W. R. Baillie, Biggings, Toab\sindex[exhibitor]{Baillie, W. R., Biggings, Toab} &
	\raggedright 4yo bull &
	\raggedright Wincote Jasper\sindex[beef]{Wincote Jasper} &
	\raggedright By Bent Caribineer; bred by Lord Nelson of Stafford; bought at Northampton for 950gns; supreme champion in 1978 \& 1982.
	\tabularnewline
\hline
	$1979$ &
	\multicolumn{4}{c|}{No entries forward.}
	\tabularnewline
\hline
	$1980$ &
	\raggedright W. R. Baillie, Biggings, Toab\sindex[exhibitor]{Baillie, W. R., Biggings, Toab} &
	\raggedright 6yo bull &
	\raggedright Wincote Jasper\sindex[beef]{Wincote Jasper} &
	\multirow{3}{8cm}{By Bent Caribineer; bred by Lord Nelson of Stafford; bought at Northampton for 950gns; supreme champion in 1978 \& 1982.}
	\tabularnewline
\cline{1-4}
	$1981$ &
	\raggedright W. R. Baillie, Biggings, Toab\sindex[exhibitor]{Baillie, W. R., Biggings, Toab} &
	\raggedright 7yo bull &
	\raggedright Wincote Jasper\sindex[beef]{Wincote Jasper} &
	\tabularnewline
\cline{1-4}
	$1982$ &
	\raggedright W. R. Baillie, Biggings, Toab\sindex[exhibitor]{Baillie, W. R., Biggings, Toab} &
	\raggedright 8yo bull &
	\raggedright Wincote Jasper\sindex[beef]{Wincote Jasper} &
	\tabularnewline
\hline
	$1983$ &
	\raggedright W. R. Baillie, Biggings, Toab\sindex[exhibitor]{Baillie, W. R., Biggings, Toab} &
	\raggedright - &
	\raggedright - &
	\raggedright -
	\tabularnewline
\hline
	$1984$ &
	\raggedright W. R. Baillie, Biggings, Toab\sindex[exhibitor]{Baillie, W. R., Biggings, Toab} &
	\raggedright 10mo bull* &
	\raggedright Ballieston Universe\sindex[beef]{Ballieston Universe} &
	\raggedright By Wincote Jasper, out of Sway Ninette.
	\tabularnewline
\hline
	$1985$ &
	\raggedright G. T. \& S. Coghill, Muce, Birsay\sindex[exhibitor]{Coghill, G. T. \& S., Muce, Birsay} &
	\raggedright 3yo cow in calf &
	\raggedright Mornview Tamar\sindex[beef]{Mornview Tamar} &
	\raggedright By Tattenhall Hublot, out of a Mornview cow.
	\tabularnewline
\hline
	$1986$ &
	\raggedright W. G. Kirkpatrick \& Sons, Newhall, Stromness\sindex[exhibitor]{Kirkpatrick, W. G. \& Sons, Newhall, Stromness} &
	\raggedright 3yo bull &
	\raggedright Ballieston Universe\sindex[beef]{Ballieston Universe} &
	\raggedright By Wincote Jasper, out of Sway Ninette.
	\tabularnewline
\hline
	$1987$ &
	\raggedright G. T. \& S. Coghill, Muce, Birsay\sindex[exhibitor]{Coghill, G. T. \& S., Muce, Birsay} &
	\raggedright 8mo bull &
	\raggedright Hekra Challenger\sindex[beef]{Hekra Challenger} &
	\raggedright By Fairthwaite Magic, out of Hekra Vera.
	\tabularnewline
\hline
	$1988$ &
	\raggedright W. R. Baillie, Biggings, Toab\sindex[exhibitor]{Baillie, W. R., Biggings, Toab} &
	\raggedright 4yo cow* &
	\raggedright Ballieston Vaila\sindex[beef]{Ballieston Vaila} &
	\raggedright By Lunsdale Harvest, out of Ballieston Snowflake.
	\tabularnewline
\hline
	$1989$ &
	\raggedright J. M. Lennie \& Co, Nearhouse, Tankerness\sindex[exhibitor]{Lennie, J. M. \& Co, Nearhouse, Tankerness} &
	\raggedright 5yo cow in calf &
	\raggedright Little Gill Verity\sindex[beef]{Little Gill Verity} &
	\raggedright By Tatton Hall Hublot, out of Hole Moor Rebecca.
	\tabularnewline
\hline
	$1990$ &
	\raggedright D. W. Sinclair \& Co, Unigarth, Sandwick\sindex[exhibitor]{} &
	\raggedright 2yo bull &
	\raggedright Royston Democrat\sindex[beef]{Royston Democrat} &
	\raggedright By Glenleary Attenhall.
	\tabularnewline
\hline
	$1991$ &
	\raggedright J. M. Lennie \& Co, Nearhouse, Tankerness\sindex[exhibitor]{Lennie, J. M. \& Co, Nearhouse, Tankerness} &
	\raggedright 7yo cow &
	\raggedright Balthayock Violet\sindex[beef]{Balthayock Violet} &
	\raggedright By Balthayock Oliver.
	\tabularnewline
\hline
	$1992$ &
	\raggedright J. M. Lennie \& Co, Nearhouse, Tankerness\sindex[exhibitor]{Lennie, J. M. \& Co, Nearhouse, Tankerness} &
	\raggedright 9yo cow in milk &
	\raggedright Little Gill Verity\sindex[beef]{Little Gill Verity} &
	\raggedright By Tattenhall Hiblot, out of Holehouse Rebecca.
	\tabularnewline
\hline
	$1993$ &
	\raggedright W. R. Baillie, Biggings, Toab\sindex[exhibitor]{Baillie, W. R., Biggings, Toab} &
	\raggedright 18mo heifer* &
	\raggedright Ballieston Helga\sindex[beef]{Ballieston Helga} &
	\raggedright By Ballieston Donald, out of Ballieston Catherine.
	\tabularnewline
\hline
	$1994$ &
	\raggedright W. R. Baillie, Biggings, Toab\sindex[exhibitor]{Baillie, W. R., Biggings, Toab} &
	\raggedright 18mo bull* &
	\raggedright Ballieston Howard\sindex[beef]{Ballieston Howard} &
	\raggedright By Cockrington Baron, out of Ballieston Frances.
	\tabularnewline
\hline
	$1995$ &
	\raggedright W. R. Baillie, Biggings, Toab\sindex[exhibitor]{Baillie, W. R., Biggings, Toab} &
	\raggedright 15mo heifer* &
	\raggedright Ballieston Jacqueline\sindex[beef]{Ballieston Jacqueline} &
	\multirow{2}{8cm}{By Fleets Rascal, out of Ballieston Donna.}
	\tabularnewline
\cline{1-4}
	$1996$ &
	\raggedright W. R. Baillie, Biggings, Toab\sindex[exhibitor]{Baillie, W. R., Biggings, Toab} &
	\raggedright 20mo heifer* &
	\raggedright Ballieston Jacqueline\sindex[beef]{Ballieston Jacqueline} &
	\tabularnewline
\hline
	$1997$ &
	\raggedright J. A. Scott, Howe, Harray\sindex[exhibitor]{Scott, J. A., Howe, Harray} &
	\raggedright 6yo bull &
	\raggedright Seawell Goutte\sindex[beef]{Seawell Goutte} &
	\raggedright By Seawell Caribbean; bought at Perth as a 13mo.
	\tabularnewline
\hline
	$1998$ &
	\raggedright G. T. \& S. Coghill, Muce, Birsay\sindex[exhibitor]{Coghill, G. T. \& S., Muce, Birsay} &
	\raggedright 5yo cow in milk &
	\raggedright New Moan Iris\sindex[beef]{New Moan Iris} &
	\raggedright By Balthayoak Dalesman, out of New Moan Emma; bought from Jim Isbister, New Moan, Harray, in October 1997.
	\tabularnewline
\hline
	$1999$ &
	\raggedright Garson Farms, Sandwick\sindex[exhibitor]{Garson Farms, Sandwick} &
	\raggedright 6yo cow* &
	\raggedright Garson Isla\sindex[beef]{Garson Isla} &
	\raggedright By Brampton Consort, out of Gredington Caniche.
	\tabularnewline
\hline
	$2000$ &
	\raggedright W. R. Baillie, Biggings, Toab\sindex[exhibitor]{Baillie, W. R., Biggings, Toab} &
	\raggedright 7yo dry cow* &
	\raggedright Ballieston Ingrid\sindex[beef]{Ballieston Ingrid} &
	\raggedright By Maerdy Victorious, out of Ballieston Ursula.
	\tabularnewline
\hline
	$2001$ &
	\multicolumn{4}{c|}{No Show due to Foot \& Mouth outbreak.}
	\tabularnewline
\hline
	$2002$ &
	\raggedright L. Spence, Spengar, Sandwick\sindex[exhibitor]{Spence, L., Spengar, Sandwick} &
	\raggedright 16mo bull* &
	\raggedright Spengar Sambhar\sindex[beef]{Spengar Sambhar} &
	\raggedright By Brampton Four Star, out of Mortimer's Lauren.
	\tabularnewline
\hline
	$2003$ &
	\raggedright Garson Farms, Sandwick\sindex[exhibitor]{Garson Farms, Sandwick} &
	\raggedright 1yo heifer &
	\raggedright Airdlin Treasure\sindex[beef]{Airdlin Treasure} &
	\raggedright By Airdlin Ploughman, out of Maerdy Limpet; bought at Perth February 2003.
	\tabularnewline
\hline
	$2004$ &
	\raggedright W. R. Baillie, Biggings, Toab\sindex[exhibitor]{Baillie, W. R., Biggings, Toab} &
	\raggedright 17mo heifer* &
	\raggedright Baillieston Topsy\sindex[beef]{Baillieston Topsy} &
	\raggedright By Moyness President, out of Ballieston Maiden.
	\tabularnewline
\hline
	$2005$ &
	\raggedright Messrs Davidson, Skaill, Sandwick\sindex[exhibitor]{Davidson, Messrs, Skaill, Sandwick} &
	\raggedright 16mo bull* &
	\raggedright Skaill Victor\sindex[beef]{Skaill Victor} &
	\raggedright By Offenbach, out of Rosette; supreme champion 2005.
	\tabularnewline
\hline
	$2006$ &
	\raggedright Messrs Davidson, Skaill, Sandwick\sindex[exhibitor]{Davidson, Messrs, Skaill, Sandwick} &
	\raggedright 16mo heifer* &
	\raggedright Skaill Abbie\sindex[beef]{Skaill Abbie} &
	\raggedright By Nectragerc, out of Perche; supreme champion, 2006; supreme champion at 2006 Dounby Show.
	\tabularnewline
\hline
	$2007$ &
	\multicolumn{4}{c|}{No Show due to Foot \& Mouth outbreak.}
	\tabularnewline
\hline
	$2008$ &
	\raggedright W. R. Baillie, Biggings, Toab\sindex[exhibitor]{Baillie, W. R., Biggins, Toab} &
	\raggedright 2yo heifer* &
	\raggedright Baillieston Anthea\sindex[beef]{Baillieston Anthea} &
	\raggedright By Mowbraypark Umpire, out of Baillieston Umbreta; supreme champion.
	\tabularnewline
\hline
	$2009$ &
	\raggedright W. R. Baillie, Biggings, Toab\sindex[exhibitor]{Baillie, W. R., Biggins, Toab} &
	\raggedright 5yo cow-in-milk* &
	\raggedright Baillieston Vera\sindex[beef]{Baillieston Anthea} &
	\raggedright By Moyness President, out of Baillieston Ola; supreme champion.
	\tabularnewline
\hline
	$2010$ &
	\raggedright J. M. Lennie \& Co, Nearhouse, Tankerness\sindex[exhibitor]{Lennie, J. M. \& Co, Nearhouse, Tankerness} &
	\raggedright 4yo bull* &
	\raggedright Nearhouse Bobby &
	\raggedright By Fremch bull Oscar, out of Brymire Ola.
	\tabularnewline
\hline
	$2011$ &
	\raggedright  &
	\raggedright  &
	\raggedright  &
	\raggedright 
	\tabularnewline
\hline
\end{longtable}

\normalsize

		\chapter{Beef Cattle}
\section[Supreme Champion]{'Hope Show Supreme Champion $1970 - 2009$}
\small
\begin{longtable}{|c|p{5.2cm}|p{3cm}|p{3cm}|p{8cm}|}
\hline
	\textbf{Year} &
	\textbf{Exhibitor} &
	\textbf{Champion} &
	\textbf{Name} &
	\textbf{Notes} 
	\tabularnewline
\hline
\endhead
	$1978$ &
	\raggedright W. R. Baillie, Biggings, Toab\sindex[exhibitor]{Baillie, W. R., Biggings, Toab} &
	\raggedright 4yo bull &
	\raggedright Wincote Jasper\sindex[beef]{Wincote Jasper} &
	\raggedright By Bent Caribineer; bred by Lord Nelson of Stafford; bought at Northampton for 950gns; supreme champion in 1978 \& 1982.
	\tabularnewline
\hline
	$1979$ &
	\multicolumn{4}{c|}{No entries forward.}
	\tabularnewline
\hline
	$1980$ &
	\raggedright W. R. Baillie, Biggings, Toab\sindex[exhibitor]{Baillie, W. R., Biggings, Toab} &
	\raggedright 6yo bull &
	\raggedright Wincote Jasper\sindex[beef]{Wincote Jasper} &
	\multirow{3}{8cm}{By Bent Caribineer; bred by Lord Nelson of Stafford; bought at Northampton for 950gns; supreme champion in 1978 \& 1982.}
	\tabularnewline
\cline{1-4}
	$1981$ &
	\raggedright W. R. Baillie, Biggings, Toab\sindex[exhibitor]{Baillie, W. R., Biggings, Toab} &
	\raggedright 7yo bull &
	\raggedright Wincote Jasper\sindex[beef]{Wincote Jasper} &
	\tabularnewline
\cline{1-4}
	$1982$ &
	\raggedright W. R. Baillie, Biggings, Toab\sindex[exhibitor]{Baillie, W. R., Biggings, Toab} &
	\raggedright 8yo bull &
	\raggedright Wincote Jasper\sindex[beef]{Wincote Jasper} &
	\tabularnewline
\hline
	$1983$ &
	\raggedright W. R. Baillie, Biggings, Toab\sindex[exhibitor]{Baillie, W. R., Biggings, Toab} &
	\raggedright - &
	\raggedright - &
	\raggedright -
	\tabularnewline
\hline
	$1984$ &
	\raggedright W. R. Baillie, Biggings, Toab\sindex[exhibitor]{Baillie, W. R., Biggings, Toab} &
	\raggedright 10mo bull* &
	\raggedright Ballieston Universe\sindex[beef]{Ballieston Universe} &
	\raggedright By Wincote Jasper, out of Sway Ninette.
	\tabularnewline
\hline
	$1985$ &
	\raggedright G. T. \& S. Coghill, Muce, Birsay\sindex[exhibitor]{Coghill, G. T. \& S., Muce, Birsay} &
	\raggedright 3yo cow in calf &
	\raggedright Mornview Tamar\sindex[beef]{Mornview Tamar} &
	\raggedright By Tattenhall Hublot, out of a Mornview cow.
	\tabularnewline
\hline
	$1986$ &
	\raggedright W. G. Kirkpatrick \& Sons, Newhall, Stromness\sindex[exhibitor]{Kirkpatrick, W. G. \& Sons, Newhall, Stromness} &
	\raggedright 3yo bull &
	\raggedright Ballieston Universe\sindex[beef]{Ballieston Universe} &
	\raggedright By Wincote Jasper, out of Sway Ninette.
	\tabularnewline
\hline
	$1987$ &
	\raggedright G. T. \& S. Coghill, Muce, Birsay\sindex[exhibitor]{Coghill, G. T. \& S., Muce, Birsay} &
	\raggedright 8mo bull &
	\raggedright Hekra Challenger\sindex[beef]{Hekra Challenger} &
	\raggedright By Fairthwaite Magic, out of Hekra Vera.
	\tabularnewline
\hline
	$1988$ &
	\raggedright W. R. Baillie, Biggings, Toab\sindex[exhibitor]{Baillie, W. R., Biggings, Toab} &
	\raggedright 4yo cow* &
	\raggedright Ballieston Vaila\sindex[beef]{Ballieston Vaila} &
	\raggedright By Lunsdale Harvest, out of Ballieston Snowflake.
	\tabularnewline
\hline
	$1989$ &
	\raggedright J. M. Lennie \& Co, Nearhouse, Tankerness\sindex[exhibitor]{Lennie, J. M. \& Co, Nearhouse, Tankerness} &
	\raggedright 5yo cow in calf &
	\raggedright Little Gill Verity\sindex[beef]{Little Gill Verity} &
	\raggedright By Tatton Hall Hublot, out of Hole Moor Rebecca.
	\tabularnewline
\hline
	$1990$ &
	\raggedright D. W. Sinclair \& Co, Unigarth, Sandwick\sindex[exhibitor]{} &
	\raggedright 2yo bull &
	\raggedright Royston Democrat\sindex[beef]{Royston Democrat} &
	\raggedright By Glenleary Attenhall.
	\tabularnewline
\hline
	$1991$ &
	\raggedright J. M. Lennie \& Co, Nearhouse, Tankerness\sindex[exhibitor]{Lennie, J. M. \& Co, Nearhouse, Tankerness} &
	\raggedright 7yo cow &
	\raggedright Balthayock Violet\sindex[beef]{Balthayock Violet} &
	\raggedright By Balthayock Oliver.
	\tabularnewline
\hline
	$1992$ &
	\raggedright J. M. Lennie \& Co, Nearhouse, Tankerness\sindex[exhibitor]{Lennie, J. M. \& Co, Nearhouse, Tankerness} &
	\raggedright 9yo cow in milk &
	\raggedright Little Gill Verity\sindex[beef]{Little Gill Verity} &
	\raggedright By Tattenhall Hiblot, out of Holehouse Rebecca.
	\tabularnewline
\hline
	$1993$ &
	\raggedright W. R. Baillie, Biggings, Toab\sindex[exhibitor]{Baillie, W. R., Biggings, Toab} &
	\raggedright 18mo heifer* &
	\raggedright Ballieston Helga\sindex[beef]{Ballieston Helga} &
	\raggedright By Ballieston Donald, out of Ballieston Catherine.
	\tabularnewline
\hline
	$1994$ &
	\raggedright W. R. Baillie, Biggings, Toab\sindex[exhibitor]{Baillie, W. R., Biggings, Toab} &
	\raggedright 18mo bull* &
	\raggedright Ballieston Howard\sindex[beef]{Ballieston Howard} &
	\raggedright By Cockrington Baron, out of Ballieston Frances.
	\tabularnewline
\hline
	$1995$ &
	\raggedright W. R. Baillie, Biggings, Toab\sindex[exhibitor]{Baillie, W. R., Biggings, Toab} &
	\raggedright 15mo heifer* &
	\raggedright Ballieston Jacqueline\sindex[beef]{Ballieston Jacqueline} &
	\multirow{2}{8cm}{By Fleets Rascal, out of Ballieston Donna.}
	\tabularnewline
\cline{1-4}
	$1996$ &
	\raggedright W. R. Baillie, Biggings, Toab\sindex[exhibitor]{Baillie, W. R., Biggings, Toab} &
	\raggedright 20mo heifer* &
	\raggedright Ballieston Jacqueline\sindex[beef]{Ballieston Jacqueline} &
	\tabularnewline
\hline
	$1997$ &
	\raggedright J. A. Scott, Howe, Harray\sindex[exhibitor]{Scott, J. A., Howe, Harray} &
	\raggedright 6yo bull &
	\raggedright Seawell Goutte\sindex[beef]{Seawell Goutte} &
	\raggedright By Seawell Caribbean; bought at Perth as a 13mo.
	\tabularnewline
\hline
	$1998$ &
	\raggedright G. T. \& S. Coghill, Muce, Birsay\sindex[exhibitor]{Coghill, G. T. \& S., Muce, Birsay} &
	\raggedright 5yo cow in milk &
	\raggedright New Moan Iris\sindex[beef]{New Moan Iris} &
	\raggedright By Balthayoak Dalesman, out of New Moan Emma; bought from Jim Isbister, New Moan, Harray, in October 1997.
	\tabularnewline
\hline
	$1999$ &
	\raggedright Garson Farms, Sandwick\sindex[exhibitor]{Garson Farms, Sandwick} &
	\raggedright 6yo cow* &
	\raggedright Garson Isla\sindex[beef]{Garson Isla} &
	\raggedright By Brampton Consort, out of Gredington Caniche.
	\tabularnewline
\hline
	$2000$ &
	\raggedright W. R. Baillie, Biggings, Toab\sindex[exhibitor]{Baillie, W. R., Biggings, Toab} &
	\raggedright 7yo dry cow* &
	\raggedright Ballieston Ingrid\sindex[beef]{Ballieston Ingrid} &
	\raggedright By Maerdy Victorious, out of Ballieston Ursula.
	\tabularnewline
\hline
	$2001$ &
	\multicolumn{4}{c|}{No Show due to Foot \& Mouth outbreak.}
	\tabularnewline
\hline
	$2002$ &
	\raggedright L. Spence, Spengar, Sandwick\sindex[exhibitor]{Spence, L., Spengar, Sandwick} &
	\raggedright 16mo bull* &
	\raggedright Spengar Sambhar\sindex[beef]{Spengar Sambhar} &
	\raggedright By Brampton Four Star, out of Mortimer's Lauren.
	\tabularnewline
\hline
	$2003$ &
	\raggedright Garson Farms, Sandwick\sindex[exhibitor]{Garson Farms, Sandwick} &
	\raggedright 1yo heifer &
	\raggedright Airdlin Treasure\sindex[beef]{Airdlin Treasure} &
	\raggedright By Airdlin Ploughman, out of Maerdy Limpet; bought at Perth February 2003.
	\tabularnewline
\hline
	$2004$ &
	\raggedright W. R. Baillie, Biggings, Toab\sindex[exhibitor]{Baillie, W. R., Biggings, Toab} &
	\raggedright 17mo heifer* &
	\raggedright Baillieston Topsy\sindex[beef]{Baillieston Topsy} &
	\raggedright By Moyness President, out of Ballieston Maiden.
	\tabularnewline
\hline
	$2005$ &
	\raggedright Messrs Davidson, Skaill, Sandwick\sindex[exhibitor]{Davidson, Messrs, Skaill, Sandwick} &
	\raggedright 16mo bull* &
	\raggedright Skaill Victor\sindex[beef]{Skaill Victor} &
	\raggedright By Offenbach, out of Rosette; supreme champion 2005.
	\tabularnewline
\hline
	$2006$ &
	\raggedright Messrs Davidson, Skaill, Sandwick\sindex[exhibitor]{Davidson, Messrs, Skaill, Sandwick} &
	\raggedright 16mo heifer* &
	\raggedright Skaill Abbie\sindex[beef]{Skaill Abbie} &
	\raggedright By Nectragerc, out of Perche; supreme champion, 2006; supreme champion at 2006 Dounby Show.
	\tabularnewline
\hline
	$2007$ &
	\multicolumn{4}{c|}{No Show due to Foot \& Mouth outbreak.}
	\tabularnewline
\hline
	$2008$ &
	\raggedright W. R. Baillie, Biggings, Toab\sindex[exhibitor]{Baillie, W. R., Biggins, Toab} &
	\raggedright 2yo heifer* &
	\raggedright Baillieston Anthea\sindex[beef]{Baillieston Anthea} &
	\raggedright By Mowbraypark Umpire, out of Baillieston Umbreta; supreme champion.
	\tabularnewline
\hline
	$2009$ &
	\raggedright W. R. Baillie, Biggings, Toab\sindex[exhibitor]{Baillie, W. R., Biggins, Toab} &
	\raggedright 5yo cow-in-milk* &
	\raggedright Baillieston Vera\sindex[beef]{Baillieston Anthea} &
	\raggedright By Moyness President, out of Baillieston Ola; supreme champion.
	\tabularnewline
\hline
	$2010$ &
	\raggedright J. M. Lennie \& Co, Nearhouse, Tankerness\sindex[exhibitor]{Lennie, J. M. \& Co, Nearhouse, Tankerness} &
	\raggedright 4yo bull* &
	\raggedright Nearhouse Bobby &
	\raggedright By Fremch bull Oscar, out of Brymire Ola.
	\tabularnewline
\hline
	$2011$ &
	\raggedright  &
	\raggedright  &
	\raggedright  &
	\raggedright 
	\tabularnewline
\hline
\end{longtable}

\normalsize

	
		%back to normal width for bibliography and indices
		\addtolength{\textwidth}{-3cm}
		\addtolength{\oddsidemargin}{+1.5cm}

		%bibliography{}
		\printindex*

\end{document}
